\section{Tmux}\label{sec:tmux}
% --------------------------------------------------------------------------------------------------------------
Tmux es un multiplexor de terminal. Es como tener pestañas de su shell sin necesidad de usar un entorno gráfico.
Me viene bien, por ejemplo, cuando accedo por SSH a una de mis Raspberry Pi 2 y deseo tener ``pestañas''.
Existen otros multiplexores de terminal, como, por ejemplo, Terminator, pero me gusta más Tmux.

La web que seguí como tutorial básico para aprender a usar Tmux es
\url{http://lukaszwrobel.pl/blog/tmux-tutorial-split-terminal-windows-easily}. Luego, estuve copiando ciertas
cosas de \url{https://bitbucket.org/jasonwryan/shiv/src/f0b3396de2bd?at=default} para añadir cosas a mi fichero
de configuración y que el banner de Tmux quede más bonito.

El fichero \path{~/.tmux.conf} es el de configuración de Tmux para un usuario en particular. En lugar de éste,
podría ser \path{~/.tmux/.conf}; vale cualquiera de los dos. Para la configuración de todo el sistema
(\foreignlanguage{english}{\textit{systemwide}}), el fichero es \path{/etc/tmux.conf} (advierta que éste no
empieza por punto). En principio, tras instalar Tmux, no tendrá estos ficheros, deberá crearlos si desea
tenerlos. Como es habitual, en caso de que existan ambos y haya discrepancias, tendrá prioridad el del usuario.
Mi preferencia actualmente es la de usar configuraciones para cada usuario. Suelo tener en repositorios de
Github mis ficheros de configuración y es bastante cómodo descargarlas.

El fichero \path{/etc/tmux.conf} que suelo usar ahora mismo en mis sistemas contiene lo siguiente:

\VerbatimInput{/etc/tmux.conf}

\noindent Para los sistemas en los que tengo habilitada la cuenta de root, creo el fichero
\path{/root/.tmux.conf}, que contiene lo siguiente:

\begin{lstlisting}[gobble=2,language=bash,style=bashinteract,escapechar=!]
  set-window-option -g window-status-current-fg red
\end{lstlisting}

\noindent Esto hace que, cuando el usuario sea root, la ``pestaña'' resaltada, es decir, la activa, en lugar de
estar coloreada de verde lo esté de rojo. Aun así, hay que evitar el uso de la cuenta root, ya que no deja
rastro en los logs del sistema. El mantenimiento, si el sistema está bien configurado, puede realizarse con
cuentas de administrador con los permisos adecuados.

Antes lo usaba, pero ahora creo que me evito problemas si lo que uso son ventanas de terminal. Tmux me ha dado
problemas con Vim, puesto que no es del todo compatible si usas xterm-256color; aquí lo explican:
\url{http://sunaku.github.io/vim-256color-bce.html}. Además, quiero dejar de usar tantas pestañas, tanto en el
shell como en Vim, y usar como buffers.
