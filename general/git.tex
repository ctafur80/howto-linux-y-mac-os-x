\section{Configurar Git}\label{sec:git}
% -------------------------------------------------------------------------------------------------------------
Antes de nada, conviene especificar que Git no es la herramienta perfecta para almacenarlo todo en la nube. Es
decir, es una herramienta increíblemente útil para crear proyectos de software de forma colaborativa o para que
gente ajena a la empresa que crea cierto software pueda contribuir dando ideas o corrigiendo errores de software
libre u open-source. Pero, por ejemplo, no es la herramienta ideal para almacenar muchos de sus ficheros
personales. La razón de esto es la comodidad: no es cómodo tener que hacer un commit por línea de comandos cada
vez que desea subir una foto a la nube. Para eso hay otras herramientas más adecuadas, como Dropbox, Google
Drive, etc. Aun así, sí es cierto que mucha gente lo usa para tener un repositorio de ciertos ficheros de
configuración de su sistema, como, por ejemplo, los de configuración de Bash o de Vim.

Si en su shell de Linux o de Mac OS X introduce

\begin{lstlisting}[gobble=2,language=bash,style=bashinteract,escapechar=!]
  !\promptu! @@man gittutorial@@
\end{lstlisting}

\noindent podrá leer un buen tutorial sobre lo básico de Git. Todo lo básico que deseaba explicar aquí, está
escrito en el tutorial.

Para actualizar el repositorio desde los cambios hechos en local, deberé introducir los siguientes tres
comandos, estando en la ruta adecuada:


\begin{lstlisting}[gobble=2,language=bash,style=bashinteract,escapechar=!]
  $ git add .
\end{lstlisting}

\begin{lstlisting}[gobble=2,language=bash,style=bashinteract,escapechar=!]
  $ git commit
\end{lstlisting}

\begin{lstlisting}[gobble=2,language=bash,style=bashinteract,escapechar=!]
  $ git push
\end{lstlisting}
