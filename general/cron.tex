\section{Programar tareas con el demonio Cron}\label{sec:cron}
% --------------------------------------------------------------------------------------------------------------
Algo que solía hacer en Linux era programar, con el uso del demonio Cron, la copia en local de los ficheros
importantes de mi sistema. Luego, pasé a usar otras herramientas, como, Déjà Dup en Ubuntu (vea
\nameref{subsec:deja-dup}) o Time Machine en Mac OS X. No obstante, dejo aquí la explicación de cómo usaba Cron,
pues puede venir bien si deseo automatizar otras tareas o si tengo que usarlo en algún sistema antiguo en el
trabajo. Creo que Cron es una herramienta que quedará obsoleta cuando se imponga Systemd en el mundo de Linux,
dentro de poco.

Lo que hacía era guardar, en el directorio \path{~/Backups}, copias de seguridad diarias de los ficheros
\path{k-yo} y \path{cecom}. Este directorio debe tenerme a mí como propietario y estar en mi grupo; estas dos
cosas se consiguen con los comandos \lstinline+chown+ y \lstinline+chgroup+ (o, simplemente, teniendo cuidado de
no crearlos como root, sino con mi cuenta principal de usuario).

En su shell introduzca el comando:

\begin{lstlisting}[gobble=2,language=bash,style=bashinteract,escapechar=!]
  !\promptu! @@crontab -e@@
\end{lstlisting}

\noindent Esto edita el fichero de configuracion de Cron del usuario que lo invoca. Creo que le pregunta, la
primera vez que lo ejecuta, qué editor desea usar. Luego, a la hora de editar ese fichero, introduzca lo
siguiente:

\begin{lstlisting}[gobble=2,language=bash,style=bashinteract,escapechar=!]
  10 2 * * 1 rm -rf /home/zfur/backups/k-yo-lun
  15 2 * * 1 cp /home/zfur/Dropbox/Documentos/otros/k-yo  /home/zfur/backups/k-yo-lun
  10 2 * * 2 rm -rf /home/zfur/Backups/k-yo-mar
  15 2 * * 2 cp /home/zfur/Dropbox/Documentos/Otros/k-yo  /home/zfur/Backups/k-yo-mar
  10 2 * * 3 rm -rf /home/zfur/Backups/k-yo-mie
  15 2 * * 3 cp /home/zfur/Dropbox/Documentos/Otros/k-yo  /home/zfur/Backups/k-yo-mie
  10 2 * * 4 rm -rf /home/zfur/Backups/k-yo-jue
  15 2 * * 4 cp /home/zfur/Dropbox/Documentos/Otros/k-yo  /home/zfur/Backups/k-yo-jue
  10 2 * * 5 rm -rf /home/zfur/Backups/k-yo-vie
  15 2 * * 5 cp /home/zfur/Dropbox/Documentos/Otros/k-yo  /home/zfur/Backups/k-yo-vie
  10 2 * * 6 rm -rf /home/zfur/Backups/k-yo-sab
  15 2 * * 6 cp /home/zfur/Dropbox/Documentos/Otros/k-yo  /home/zfur/Backups/k-yo-sab
  10 2 * * 0 rm -rf /home/zfur/Backups/k-yo-dom
  15 2 * * 0 cp /home/zfur/Dropbox/Documentos/Otros/k-yo  /home/zfur/Backups/k-yo-dom

  20 2 * * 1 rm -rf /home/zfur/Backups/cecom-lun
  25 2 * * 1 cp /home/zfur/Dropbox/Documentos/Importante/cecom /home/zfur/Backups/cecom-lun
  20 2 * * 1 rm -rf /home/zfur/Backups/cecom-mar
  25 2 * * 1 cp /home/zfur/Dropbox/Documentos/Importante/cecom /home/zfur/Backups/cecom-mar
  20 2 * * 1 rm -rf /home/zfur/Backups/cecom-mie
  25 2 * * 1 cp /home/zfur/Dropbox/Documentos/Importante/cecom /home/zfur/Backups/cecom-mie
  20 2 * * 1 rm -rf /home/zfur/Backups/cecom-jue
  25 2 * * 1 cp /home/zfur/Dropbox/Documentos/Importante/cecom /home/zfur/Backups/cecom-jue
  20 2 * * 1 rm -rf /home/zfur/Backups/cecom-vie
  25 2 * * 1 cp /home/zfur/Dropbox/Documentos/Importante/cecom /home/zfur/Backups/cecom-vie
  20 2 * * 1 rm -rf /home/zfur/Backups/cecom-sab
  25 2 * * 1 cp /home/zfur/Dropbox/Documentos/Importante/cecom /home/zfur/Backups/cecom-sab
  20 2 * * 1 rm -rf /home/zfur/Backups/cecom-dom
  25 2 * * 1 cp /home/zfur/Dropbox/Documentos/Importante/cecom /home/zfur/Backups/cecom-dom
\end{lstlisting}

\noindent En este ejemplo, se ha programado Cron para que a las 2:10 am se elimine el fichero
\path{k-<día-semana>} del directorio \path{~/Backups} y 5 minutos después haga una copia del mismo
fichero. Lo mismo con el fichero \path{cecom}, pero a otras horas.
