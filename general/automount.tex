\section{Automontar}\label{sec:automontar}
% --------------------------------------------------------------------------------------------------------------
Cuando aquí hablamos de \emph{unidades}, nos referimos tanto a medios removibles de almacenamiento, como a
teléfonos o reproductores de medios que hagan uso del protocolo media transfer protocol (MTP), como a cámaras de
foto digitales, etc.

Esta sección le interesa si no va a instalar un entorno de escritorio en su sistema. Los entornos de escritorio,
al menos los más usados, le instalan sin pedirle permiso herramientas que configuran el automontado de unidades
en su sistema. Por ejemplo, si tiene un entorno de escritorio Gnome o uno basado en Gnome, tendrá instalado
Gnome Virtual File System (\lstinline!gvfs!), que se encarga de automontar unidades en su sistema y será usado
por su file manager (gestor de ficheros), como Nautilus, por ejemplo, o el que sea que tenga.

Lo primero que debe hacer es instalar el paquete \lstinline!udisks2!. Ahora, cree el fichero
\path{/etc/polkit-1/rules.d/50-udisks.rules} con el siguiente contenido:

\begin{lstlisting}[gobble=2,style=bashcode,escapechar=!]
  polkit.addRule(function(action, subject) {
    var YES = polkit.Result.YES;
    var permission = {
      //// only required for udisks1:
      //"org.freedesktop.udisks.filesystem-mount": YES,
      //"org.freedesktop.udisks.filesystem-mount-system-internal": YES,
      //"org.freedesktop.udisks.luks-unlock": YES,
      //"org.freedesktop.udisks.drive-eject": YES,
      //"org.freedesktop.udisks.drive-detach": YES,
      // only required for udisks2:
      "org.freedesktop.udisks2.filesystem-mount": YES,
      "org.freedesktop.udisks2.filesystem-mount-system": YES,
      "org.freedesktop.udisks2.encrypted-unlock": YES,
      "org.freedesktop.udisks2.eject-media": YES,
      "org.freedesktop.udisks2.power-off-drive": YES,
      // required for udisks2 if using udiskie from another seat (e.g. systemd):
      "org.freedesktop.udisks2.filesystem-mount-other-seat": YES,
      "org.freedesktop.udisks2.encrypted-unlock-other-seat": YES,
      "org.freedesktop.udisks2.eject-media-other-seat": YES,
      "org.freedesktop.udisks2.power-off-drive-other-seat": YES
    };
    if (subject.isInGroup("storage")) {
      return permission[action.id];
    }
  });
\end{lstlisting}

\noindent Yo, como suelo usar \lstinline!udisks2!, en este último listado comento (con \lstinline+//+, ya que
este fichero tiene código Javascript) la parte de \lstinline!udisks1!. Esto hace que los usuarios que pertenecen
al grupo \texttt{storage} puedan montar sistemas de ficheros. Tras reiniciar el sistema, ahora debería ser capaz
de montar discos externos con cualquier usuario del grupo \texttt{storage}. Si lo ha probado con el comando
\lstinline!mount! y no le ha funcionado, es porque debería haber usado el comando de Udisks2 para montar, que
es:

\begin{lstlisting}[gobble=2,language=bash,style=bashinteract,escapechar=!]
  !\promptu! @@udisksctl mount -b !\val{disp-bloque}!@@
\end{lstlisting}

Ahora deberá instalar \lstinline!udevil!, que es un paquete que incluye el software \lstinline!devmon!, que es
un Udisks wrapper, una herramienta que permite el automontado de unidades.

Ahora, para montar unidades con \lstinline!udisks! o \lstinline!udisks2!, elimine el permiso SUID de
\lstinline!udevil!:

\begin{lstlisting}[gobble=2,language=bash,style=bashinteract,escapechar=!]
  !\promptr! @@chmod -s /usr/bin/udevil@@
\end{lstlisting}

\noindent Tras esto, deberá habilitar el servicio \lstinline!devmon! para que se inicie con el inicio del
sistema:

\begin{lstlisting}[gobble=2,language=bash,style=bashinteract,escapechar=!]
  !\promptr! @@systemctl enable devmon@!\val{usuario}!.service@@
\end{lstlisting}

\noindent En mi caso, \val{usuario} es kodi. Deberá hacerlo para todos los usuarios que desee que puedan
automontar unidades.

Lo que voy a explicar ahora tiene el propósito de que las unidades que se monten automáticamente no lo hagan a
\path{/run/media/\$USER/}, que es la ruta predeterminada, sino a \path{/media/}. Para ello, deberá crear una
regla en su sistema.

Primero cree el directorio \path{/media/}, si es que no existe.

Ahora cree la regla, es decir, cree el fichero \path{/etc/udev/rules.d/99-udisks2.rules} con el siguiente
contenido:

\begin{lstlisting}[gobble=2,language=bash,style=bashcode]
  # UDISKS_FILESYSTEM_SHARED
  # ==1: mount filesystem to a shared directory (/media/VolumeName)
  # ==0: mount filesystem to a private directory (/run/media/$USER/VolumeName)
  # See udisks(8)
  ENV{ID_FS_USAGE}=="filesystem|other|crypto", ENV{UDISKS_FILESYSTEM_SHARED}="1"
\end{lstlisting}

Las distribuciones de Linux no suelen traer instalado software para montar unidades de almacenamiento con
sistema de ficheros NTFS. Para instalarlo en Arch Linux, deberá instalar el paquete \lstinline!ntfs-3g!. Para
exFAT, \lstinline!fuse-exfat! o \lstinline!exfat-utils! (creo que es preferible el primero; vea
\nameref{sec:exfat}). Para montar ISOs, puede usar \lstinline!fuseiso! (que, lógicamente, emplea FUSE).
