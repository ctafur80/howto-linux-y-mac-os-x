\section{Passwordstore}\label{sec:passwordstore}
% --------------------------------------------------------------------------------------------------------------
El software Pass (también conocido como Passwordstore) usa algún software compatible con PGP, como GnuPG (vea
\nameref{sec:gnupg}). Por lo tanto, antes de nada, debe tener una clave PGP configurada en su sistema. Al
basarse en ficheros cifrados con GnuPG, tienen extensión de nombre de fichero \path{.gpg}. Estos ficheros se
encuentran ubicados en \path{~/.password-store}, de modo que es conveniente incluir dicho directorio también en
la copia de seguridad que haga periódicamente en su sistema. También podrá tener una copia en un repositorio de
Github, tal y como se explicará aquí.

\noindent Antes de instalar Pass, necesita crear o importar un par de claves privada-pública de PGP en su
sistema (vea \nameref{sec:gnupg}), pues, sin ese par de claves, no podrá encriptar/desencriptar.

En la web \url{www.zx2c4.com/projects/password-store/} hay un manual de cómo se usa el software Pass. Más
sencillo es recurrir a la ayuda \lstinline!man! desde su shell de Mac OS X o Linux.

A continuación se expone el proceso para configurar en una máquina nueva (es decir, recién
formateada/borrada\ldots ) sus claves cifradas de Pass haciendo uso de un repositorio de Github.

Lo primero que debe hacer es crear un repositorio nuevo de Github, ya sea en la web, en aplicación, etc.

Una vez hecho esto, deberá configurar en su sistema el nombre de usuario y el e-mail de su cuenta de Github:

\begin{lstlisting}[gobble=2,language=bash,style=bashinteract,escapechar=!]
  !\promptu! @@git config --global user.name "!\val{git\_user}!"@@
  !\promptu! @@git config --global user.email "!\val{e-mail}!"@@
\end{lstlisting}

\noindent Si tiene otros repositorios en su sistema, este paso se lo puede saltar, pues lo habrá hecho en algún
otro momento anterior. En mi caso, \val{git\_user} es \lstinline!zfur! y, \val{e-mail},
\lstinline!ctafur@gmail.com!. Ahora, inicialice un nuevo almacén de contraseñas
(\foreignlanguage{english}{\textit{password store}}):

\begin{lstlisting}[gobble=2,language=bash,style=bashinteract,escapechar=!]
  !\promptu! @@pass init !\val{e-mail}!@@
\end{lstlisting}

\noindent La respuesta del comando le dirá que se ha creado el directorio \path{~/.password-store/} y que se
ha inicializado el almacén de contraseñas para \val{e-mail}.

%Ahora deberá inicializar primero un repositorio
%Github en su directorio home (\path{/home/<n\_usuario>/}), si es que no lo tenía configurado:
%
%\begin{Verbatim}[gobble=2,commandchars=\\\{\}]
%  \!\promptu! @@git init}
%\end{lstlisting}
%
%\noindent Repito, este último comando debe haberlo introducido con su prompt en el directorio home del usuario.
%El repositorio, en principio, estará vacío; sólo tendrá ficheros de configuración. El comando le dirá que el
%repositorio Github que acaba de crear se encuentra en \path{~/.git/}.
%
Ahora debe inicializar el repositorio Github que usará Pass:

\begin{lstlisting}[gobble=2,language=bash,style=bashinteract,escapechar=!]
  !\promptu! @@pass git init@@
\end{lstlisting}

\noindent Puede hacer esto con el prompt donde quiera. El repositorio, en principio, estará vacío; sólo tendrá
ficheros de configuración. El comando le dirá que el repositorio Github que acaba de crear se encuentra en
\path{~/.password-store/.git/}. Ahora deberá mover su prompt a \path{~/.password-store/.git/}:

\begin{lstlisting}[gobble=2,language=bash,style=bashinteract,escapechar=!]
  !\promptu! @@cd ~/.password-store/.git/@@
\end{lstlisting}

\noindent Esto es necesario para el siguiente paso. Ahora deberá añadir un remoto a su repositorio:

\begin{lstlisting}[gobble=2,language=bash,style=bashinteract,escapechar=!]
  !\promptu! @@git remote add !\val{local\_remo}! !\val{url\_remo}!@@
\end{lstlisting}

\noindent En mi caso, \val{local\_remo} vale \lstinline!origin! y, \val{url\_remo},
\lstinline!https://github.com/zfur/pass-store.git!. Si no tuviera su prompt en \path{~/.password-store/.git/}
antes de introducir este último comando, podría haberle dado un error o se hubiera añadido el repositorio donde
no debía. Ahora deberá verificar que se ha añadido correctamente el remoto a su repositorio; deberá introducir
el siguiente comando con su prompt en \path{~/.password-store/}:

\begin{lstlisting}[gobble=2,language=bash,style=bashinteract,escapechar=!]
  !\promptu! @@git remote -v@@
  origin  https://github.com/!\val{user}!/!\val{repo}!.git (fetch)
  origin  https://github.com/!\val{user}!/!\val{repo}!.git (push)
\end{lstlisting}

\noindent Deberá devolverle algo parecido a lo que he puesto aquí. Si lo desea, puede comprobar que, con su
prompt en \path{~/.git/}, no le devuelve esto mismo, pues ahí no tiene añadido este remoto. Ahora, por fin,
puede hacer la descarga de los datos de su remoto a su repositorio local; es decir, el
\foreignlanguage{english}{\textit{pull}} o \foreignlanguage{english}{\textit{fetch}}:

\begin{lstlisting}[gobble=2,language=bash,style=bashinteract,escapechar=!]
  !\promptu! @@pass git pull !\val{local\_remo}! !\val{branch}!@@
\end{lstlisting}

\noindent En mi caso, \val{local\_remo} es \lstinline!origin! y, \val{branch}, \lstinline!master!. Se abrirá un
editor de texto en el que se le pedirá que explique por qué ha unido este sistema a su cuenta de Github. Deberá
escribir algo, lo que sea, para que se ejecute correctamente el comando anterior; deberá introducirlo en la
primera lína, la de más arriba.

Ahora, estando en la ruta del repositorio en su ordenador, introduzca:

\begin{lstlisting}[gobble=2,language=bash,style=bashinteract,escapechar=!]
  !\promptu! @@git push --set-upstream !\val{repo}! !\val{branch}!@@
\end{lstlisting}

\noindent En mi caso, \val{repo} es \lstinline!origin! y, \val{branch}, \lstinline!master!, como antes.

Ahora, si su repositorio es antiguo, puede hacer la siguiente configuración para que Github funcione como en la
última versión:

\begin{lstlisting}[gobble=2,language=bash,style=bashinteract,escapechar=!]
  !\promptu! @@git config --global push.default simple@@
\end{lstlisting}

\noindent Esto, en concreto, lo que hace es que el push que hace si no se especifica más es uno simple.

Ahora sí puede hacer, desde la ruta adecuada del prompt, el

\begin{lstlisting}[gobble=2,language=bash,style=bashinteract,escapechar=!]
  !\promptr! @@pass git push !\val{repo}! !\val{branch}!@@
\end{lstlisting}

\noindent para que se actualice el repositorio en Internet con los cambios que le haya hecho usted en su
ordenador.

Para hacer pull o push a/de este remoto, no olvide que ha de tener su prompt en
\path{~/.password-store/.git/}.

Para eliminar un repositorio de Github que haya creado usted, deberá, simplemente, eliminar el directorio
\path{.git/} de ese repositorio. ¡Ojo!, el que se encuentra en la ruta adecuada; puede que tenga varios
directorios \path{.git/} en su sistema, por ejemplo, si tiene varios repositorios. Para saber cómo eliminar, con
su shell de Linux o de Mac OS X, un directorio de su sistema, vea \nameref{sec:elim-dir.tex}.

Para Mac OS X, puede instalar Pass mediante Homebrew. Sólo tendrá que introducir:

\begin{lstlisting}[gobble=2,language=bash,style=bashinteract,escapechar=!]
  !\promptu! @@brew install --HEAD pass@@
\end{lstlisting}

Luego tiene que inicializar su cuenta de Pass:

\begin{lstlisting}[gobble=2,language=bash,style=bashinteract,escapechar=!]
  !\promptu! @@pass init !\val{e-mail}!@@
\end{lstlisting}

\noindent etc.
