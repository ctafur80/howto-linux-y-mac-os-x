\section{Homebrew en Mac OS X}\label{sec:homebrew-m}
% -------------------------------------------------------------------------------------------------------------
Homebrew es un gestor de paquetes (\foreignlanguage{english}{\textit{package manager}}) para Mac OS X. Está
escrito completamente en el lenguaje Ruby y hace uso de Git. El sitio web oficial es \url{http://brew.sh/}. Para
instalarlo en su sistema, sólo tiene que introducir lo siguiente en su shell o emulador de terminal:

\begin{lstlisting}[gobble=2,language=bash,style=bashinteract,escapechar=!]
  !\promptu! @@ruby -e "$(curl -fsSL https://raw.githubusercontent.com/Homebrew/install/master/install)"@@
\end{lstlisting}

\noindent Esto hará que se instale bajo \path{/usr/local/} y, por tanto, que no necesite permisos de
administración para usarlo para instalar software. Quizás esto sea malo, porque creo que permite que cualquier
usuario del sistema, aunque no se administrador, pueda instalar software mediante Homebrew en el mismo.

Una vez que lo tenga instalado, si desea instalar el paquete \val{paquete}, sólo tendrá que introducir:

\begin{lstlisting}[gobble=2,language=bash,style=bashinteract,escapechar=!]
  !\promptu! @@brew install !\val{paquete}!@@
\end{lstlisting}

Aunque hay veces que deberá hacerlo de otro modo:

\begin{lstlisting}[gobble=2,language=bash,style=bashinteract,escapechar=!]
  !\promptu! @@brew install --HEAD !\val{paquete}!@@
\end{lstlisting}

El software que haya instalado mediante Homebrew residirá bajo \path{/usr/local/Cellar/} y bajo
\path{/usr/local/bin/} tendrá, junto al resto de software que haya instalado tras la instalación de Mac OS X,
vínculos simbólicos a \path{/usr/local/Cellar/}.

Puede encontrar la documentación oficial en
\url{https://github.com/Homebrew/homebrew/tree/master/share/doc/homebrew#readme}.

El software que suelo instalar mediante Homebrew es el siguiente:

\begin{itemize}
  \item Máxima prioridad:
  \begin{itemize}
    \item Wget. Para descargar de internet.
    \item Vim. Es mi editor de textos favorito y el software que más uso en mi ordenador.
    \item Git. Me permite clonar repositorios de sitios de SVC como Github u otros basados en Git. Gracias a
      esto puedo, por ejemplo, descargar muy fácilmente, tras una reinstalación de mi sistema, los ficheros de
      configuración de Bash y de Vim.
    \item Bash. Suelo descargar una versión más moderna de Bash que la que trae Mac OS, pues, hay cosas, como el
      tab autocompletion que no funcionan tan bien como desearía en las versiones antiguas.
    \item Gpg. También conocido como GnuPG.
    \item Herramientas de \TeX{} que se instalan solas con la instalación de MacTeX. Ojo, no instalo
      MacTeX con Homebrew, sino con su instalador.
  \end{itemize}
  \item Prioridad secundaria:
    \begin{itemize}
      \item Mosh. Es como SSH pero más moderno. No sé si es menos seguro, lo que sí he notado es que es más
        estable.
      \item Pass. También conocido como Passwordstore. Para almacenar claves de forma cifrada en su sistema.
      \item Python3.
      \item Screen. Para poder usar 256 colores en el emulador de terminal que uso, que es el predefinido de Mac
        OS X.
      \item Tmux. Un multiplexor de terminal. Estoy intentando no usarlo tanto.
      \item Bash completion. No sé si será necesario instalarlo. Lo veré la próxima vez que ``formatee'' el
        sistema operativo.
\end{itemize}
  \item Prioridad terciaria:
    \begin{itemize}
      \item Atom. El editor de textos GUI. Yo prefiero usar Vim bajo mi emulador de terminal.
      \item Fish. Es un shell, un sustituto de Bash que está teniendo algo de éxito entre los usuarios de Mac OS
        X, aunque también lo está teniendo Zsh y yo prefiero seguir con Bash.
    \end{itemize}
\end{itemize}

Hay cierto software que prefiero instalarlo con su propio instalador, como, por ejemplo, \TeX{} Live, que es el
software que contiene todo lo necesario para procesar documentos de \TeX{}, \LaTeX{}, ConTeXt, LuaTeX,
etc., es decir, \TeX{} y todos sus derivados. Es conveniente usar en este caso su propio instalador, puesto que
conviene tenerlo, sobretodo por los paquetes (packages). \TeX{} Live trae su propio gestor de actualización,
llamado con el comando \lstinline!tlmgr!.
