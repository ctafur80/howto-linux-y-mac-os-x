\section{Usar tarjetas SD con sistema de ficheros exFAT}\label{sec:exfat}
% --------------------------------------------------------------------------------------------------------------
Creo que lo aconsejable es usar algún formato de archivo que sepamos que funciona bien en Linux, como EXT4 o FAT
(de \num{16} o \SI{32}{\bit}), por ejemplo; aunque debe tener en cuenta que para FAT el tamaño máximo de fichero
es de \SI{4}{\gibi\byte}. Las distribuciones de Linux no traen el software necesario para escribir en
particiones con formato NTFS; para poder usarlo, deberá instalar el paquete \lstinline!ntfs-3g!, y nadie le
garantiza que le vaya a funcionar bien---aunque también es cierto que ya no suele dar problemas---. En
definitiva, si desea compatibilidad con Windows, use FAT; si desea usarlo sólo para Linux y otros sistemas UNIX,
es mejor que use EXT4, o, en un futuro, cuando sea lo bastante estable, algún otro sistema de ficheros más
moderno, como BTRFS. De todos modos, no creo que tengan mucho futuro las tarjetas SD, pues, bajo mi punto de
vista, el almacenamiento externo se hará normalmente a través de la nube cuando las redes tengan mayor capacidad
en unos años.

Si desea usar una tarjeta SD (o microSD, nanoSD, etc.) en Linux manteniendo el sistema de ficheros exFAT, que es
el que suelen traer este tipo de tarjetas---aunque en realidad pueden usar casi cualquier sistema de
ficheros---, lo que debe hacer primero es instalar el programa Fuse, pues las distribuciones de Linux no suelen
traer instalado el software para usar exFAT. Esto, en Ubuntu, se hace del modo siguiente:

\begin{enumerate}

  \item añada el repositorio \lstinline!relan/exfat!:

    \begin{lstlisting}[gobble=6,language=bash,style=bashinteract,escapechar=!]
      !\promptr! @@add-apt-repository ppa:relan/exfat@@
    \end{lstlisting}

  \item actualice la lista de paquetes e instale el paquete \lstinline+fuse-exfat+:

    \begin{lstlisting}[gobble=6,language=bash,style=bashinteract,escapechar=!]
      !\promptr! @@apt-get update@@
      !\promptr! @@apt-get install fuse-exfat@@
    \end{lstlisting}

\end{enumerate}

Una vez instalado el programa Fuse, deberá actuar según lo que desee. Si desea configurar el automontado de
dichas tarjetas, deberá configurarlo o, si usa un entorno de escritorio, se puede configurar solo, sin que tenga
que hacer nada usted (vea \nameref{sec:automontar}).
