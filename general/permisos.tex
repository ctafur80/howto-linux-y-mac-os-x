\section{Gestión de permisos}\label{sec:permisos}
% --------------------------------------------------------------------------------------------------------------
Existen tres formas de gestionar los permisos en Linux.

\subsection{Permisos de UNIX}
Es la forma más antigua de las que exponemos aquí. Es también la más basta o, como dirían en inglés, la menos
\foreignlanguage{english}{\textit{fine grained}}. Debería ser, por tanto, la primera opción para los permisos de
lectura o escritura, e incluso de ejecución de pequeños programas no muy complejos.

Los permisos de UNIX es uno de los primeros temas que se abordan en cualquier libro o tutorial de Linux. Para
cada fichero existen permisos de lectura (\lstinline!r!), escritura (\lstinline!w!) y ejecución (\lstinline!x!)
para el dueño (\lstinline!u!) del fichero, el grupo principal del fichero (\lstinline!g!) y para el resto de
usuarios (\lstinline!o!). La opción \lstinline!-l! del comando \lstinline!ls! le permite ver estos permisos. Por
ejemplo:

\begin{lstlisting}[gobble=2,language=bash,style=bashinteract,escapechar=!]
  !\promptu! @@ls -l@@
  total 96
  -rwxrwxrwx  1 zFur  staff  48391 Jan  2 23:20 The.Man.In.The.High.Castle.S01E08.WEBRip.x264-TASTETV.srt
  drwxr-xr-x  8 zFur  staff    272 Dec 25 20:35 Virtual Machines.localized
\end{lstlisting}

\noindent Los verá en la primera columna. La primera posición indica el tipo de fichero (directorio, vínculo
simbólico, etc.; si es un fichero normal, se indica con un guión (\lstinline!-!)). Luego, justo a su derecha
vienen los permisos de \lstinline!u! (recuerde, el usuario dueño del fichero), luego los de \lstinline!g! y,
finalmente, los de \lstinline!o!. Los permisos que no se tengan, aparecerán como un guión (\lstinline!-!); no lo
confunda con el guión de la columna a la izquierda del todo. Existe un particularidad que debe conocer: la
\lstinline!x!, si el ``fichero'' del que se trata es un directorio, indica si se puede (o no se puede) listar
sus contenidos. Como dije antes, la explicación de esto viene en cualquier libro de Linux, o en alguna página de
man también, seguramente, así que no se explicará en más detalle aquí.

\subsection{ACLs}
Los ACLs (\foreignlanguage{english}{\textit{access control lists}}) permiten un ajuste mayor que los permisos de
UNIX. Permite, por ejemplo, que se puedan entregar permisos de lectura de un documento en concreto a un usuario
en particular, sin necesidad de crear un grupo \textit{ad hoc} en el que sólo estaría ese usuario. No sé si el
software necesario para usar ACLs viene con la instalación básica de Arch Linux. En
\url{https://wiki.archlinux.org/index.php/Access_Control_Lists} puede ver una buena explicación del uso de de
ACLs.

Lo normal es que su sistema monte sus sistemas de ficheros con la opción \lstinline!acl! para así poder usar
ACLs, pero, si no está seguro, puede comprobarlo con:

\begin{lstlisting}[gobble=2,language=bash,style=bashinteract,escapechar=!]
  !\promptr! @@tune2fs -l /dev/sdXY | grep "Default mount options:"@@
  Default mount options:    user_xattr acl
\end{lstlisting}

Usted puede ver que un fichero de su sistema tiene ACLs porque, por ejemplo, al hacer

\begin{lstlisting}[gobble=2,language=bash,style=bashinteract,escapechar=!]
  !\promptu! @@ls -l@@
\end{lstlisting}

\noindent para ver los permisos del directorio en el que se encuentra el fichero, le aparece, justo a la derecha
de los permisos de UNIX de dicho fichero, un signo más (\lstinline!+!). Por ejemplo, podría aparecerle algo así:

\begin{lstlisting}[gobble=2,language=bash,style=bashinteract,escapechar=!]
  !\promptu! @@ls -l /dev/audio@@
  crw-rw----+ 1 root audio 14, 4 nov.   9 12:49 /dev/audio
\end{lstlisting}

\noindent Entonces, podrá ver los ACLs con el siguiente comando:

\begin{lstlisting}[gobble=2,language=bash,style=bashinteract,escapechar=!]
  !\promptu! @@getfacl /dev/audio@@
  getfacl: Removing leading '/' from absolute path names
  # file: dev/audio
  # owner: root
  # group: audio
  user::rw-
  user:solstice:rw-
  group::rw-
  mask::rw-
  other::---
\end{lstlisting}

\noindent En este caso le indica, además de los permisos de UNIX, que el usuario \texttt{solstice} tiene
permisos de lectura (\lstinline!r!) y escritura (\lstinline!w!) del fichero consultado.

Para dar permisos ACL a un usuario \val{usuario}, debe usar el comando:

\begin{lstlisting}[gobble=2,language=bash,style=bashinteract,escapechar=!]
  !\promptr! @@setfacl -m "u:!\val{usuario}!:!\val{permisos}!"@@
\end{lstlisting}

\noindent, si, en lugar de a un usuario, es a un grupo \val{grupo},

\begin{lstlisting}[gobble=2,language=bash,style=bashinteract,escapechar=!]
  !\promptr! @@setfacl -m "g:!\val{grupo}!:!\val{permisos}!"@@
\end{lstlisting}

\noindent Para eliminar de un fichero todos los permisos ACL,

\begin{lstlisting}[gobble=2,language=bash,style=bashinteract,escapechar=!]
  !\promptr! @@setfacl -b@@
\end{lstlisting}

\subsection{Polkit}\label{subsec:polkit}
Polkit es la última forma que surjió para la gestión de permisos. La idea detrás de Polkit es algo distinta a
la de los permisos de Unix y la de las ACLs. Es distinta porque en este caso el propósito es que se comuniquen
de forma segura procesos sin privilegios con procesos con más privilegios. Es decir, es un marco que centraliza
el proceso de toma de decisión respecto a entregar acceso a operaciones privilegiadas a aplicaciones sin
privilegios. No es tanto de leer/modificar ficheros, como en los casos anteriores.

Puede ver una buena explicación de Polkit en \url{https://wiki.archlinux.org/index.php/Polkit}.

En Arch Linux para arquitectura ARM no viene de serie y, por tanto, deberá instalar el paquete
\lstinline!polkit!. Polkit hace uso de agentes de autenticación para permitir al usuario de una sesión
demostrar que él es realmente el usuario de dicha sesión o que es un administrador del sistema. Hay agentes de
autenticación tanto de terminal como de GUI.
