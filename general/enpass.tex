\section{Enpass}\label{sec:enpass}
% --------------------------------------------------------------------------------------------------------------
Este gestor de claves funciona por una GUI y existe para los principales sistemas, incluidos Linux y Android.
Puedo usar mi cuenta de Dropbox o de Google Drive para sincronizar las claves entre los distintos clientes que
use, lo cual es muy cómodo, pues puedo escribir las claves en mi portátil con Ubuntu y luego, al sincronizar,
las tengo en mi móvil Android. La única limitación es que la versión gratuita de Android tiene un límite de 20
claves. La versión de pago creo que es un servicio de pago mensual; tengo que informarme. Ahora mismo, para
gestionar mis claves, uso Pass (también conocido como Passwordstore; vea \nameref{sec:passwordstore}), pero
estaría bien tenerlas también en otro gestor como Enpass por si las pierdo. Aun así, hay que tener cuidado con
este tipo de programas. Lo mejor es usar alguno que esté muy contrastado que no tiene problemas de seguridad.
