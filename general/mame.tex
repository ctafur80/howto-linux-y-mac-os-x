\section{Instalar el software emulador de videojuegos Mame}\label{sec:mame}
% --------------------------------------------------------------------------------------------------------------
El programa Mame que solía usar es el que se llama Mame en el Software Center de Ubuntu. Tiene muy buenas
opiniones en el Software Center. Una vez instalado dicho programa, conviene hacer que las ROMs las almacenemos
en nuestro directorio personal (es decir, en nuestro home). Esto se puede hacer del siguiente modo.

\begin{enumerate}

  \item Cree el directorio \path{roms/} en \path{~/.mame/}:

    \begin{lstlisting}[gobble=6,language=bash,style=bashinteract,escapechar=!]
      !\promptu! @@mkdir ~/.mame/roms@@
    \end{lstlisting}

  \item Cree luego un enlace simbólico en el directorio \path{/usr/local/share/games/mame/} y llámelo
    \path{roms}; se hace así:

    \begin{lstlisting}[gobble=6,language=bash,style=bashinteract,escapechar=!]
      !\promptr! @@ln -sd /usr/local/share/games/mame/ roms@@
    \end{lstlisting}

  \item Ahora ya puede almacenar sus ROMs en su directorio personal y no perderlas si formatea el equipo y se
    acuerda de guardar los datos de su directorio personal.

\end{enumerate}

Conviene advertir de que lo normal es que las ROMs que funcionan con nuestro programa emulador son las que
tienen la misma versión que éste, cosa que no siempre es fácil de encontrar. Aun así, creo que para jugar a
emuladores lo mejor es usar un dispositivo que usemos solo para eso. Por ejemplo, un ordenador no muy potente
puede servirle o una Raspberry Pi. El problema de la Raspberry Pi es que, aunque tenga potencia para ejecutar
los juegos (al menos, la versión 2), la mayoría de las ROMs están hechas para arquitectura x86, no para ARM. Así
que tiene que usar programas para convertirlas. En Raspberry Pi la mejor forma de instalar emuladores suele ser
usando la distribución de Linux Retropie o usando alguna de media center (como, por ejemplo, Kodi) y
posteriormente instalar algún plugin para ejecutar ROMs. Además de emuladores de máquinas recreativas (Mame),
puede encontrar también emuladores de otras plataformas como, por ejemplo, las videoconsolas de Nintendo, Sony
Playstation, etc.
