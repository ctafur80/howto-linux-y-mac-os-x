\section{Encriptar un pendrive o un archivo con LUKS/dm-crypt}\label{sec:luks}
% --------------------------------------------------------------------------------------------------------------
LUKS significa \foreignlanguage{english}{Linux Unified Key Setup}, y es un método de encriptación\footnote{Para
ver la terminología y notación que se emplea en este documento, incluida la polémica de los términos
\emph{encriptar} y \emph{cifrar}, vea \nameref{sec:notacion}.} que se implementa en Linux por medio del paquete
\lstinline!cryptsetup!, y, en Windows, por FreeOTF. Este tutorial es prácticamente una copia del que aparece en
\url{https://balau82.wordpress.com/2011/08/25/encrypting-data-on-usb-flash-drives-with-luks/}, aunque algunas
cosas las he hecho de modo distinto. Aun así, creo que Google ha creado encriptación a nivel del sistema de
ficheros EXT4 y este desarrollo lo ha liberado como software libre; creo que se implementará a partir de las
versiones 4.1 o 4.2 del núcleo de Linux.

Antes de nada, deberá tener instalada la herramienta Cryptsetup en su distribución de Linux. Para las que usan
el sistema de paquetes APT, se hará con:

\begin{lstlisting}[gobble=2,language=bash,style=bashinteract,escapechar=!]
  !\promptr! @@apt-get install cryptsetup@@
\end{lstlisting}

\noindent Para Windows, se deberá instalar la herramienta FreeOTF.

Una vez hecho esto, los pasos a seguir son los siguientes:

\begin{enumerate}

  \item Inserte el medio removible (como, por ejemplo, el pendrive) y pruebe con el comando \lstinline+lsblk+ el
    dispositivo (\foreignlanguage{english}{\textit{device}}) que se crea para acceder a éste. Lo normal es que
    sea \path{/dev/sd<x>}, siendo \val{x} una letra mayor que \texttt{a}. Si no está seguro de cuál es el
    dispositivo, no siga, pues podría perder los datos de alguna partición de alguno de sus discos duros. Para
    el caso de ejemplo que mostramos aquí, supondremos que el dispositivo es \path{/dev/sdc}. En la web original
    donde aprendí este proceso, decían que se usase el comando \lstinline+dmesg+, pero creo que es más sencillo
    de ver con \lstinline+lsblk+.

    \begin{lstlisting}[gobble=6,language=bash,style=bashinteract,escapechar=|]
      |\promptr| @@fdisk /dev/sdc@@
      Command (m for help): @@o@@
      Building a new DOS disklabel with disk identifier 0xe6978b70.
      Changes will remain in memory only, until you decide to write them.
      After that, of course, the previous content won't be recoverable.

      Warning: invalid flag 0x0000 of partition table 4 will be corrected by w(rite)

      Command (m for help): @@w@@
      The partition table has been altered!

      Calling ioctl() to re-read partition table.

      WARNING: Re-reading the partition table failed with error 16: Device or resource busy.
      The kernel still uses the old table. The new table will be used at
      the next reboot or after you run partprobe(8) or kpartx(8)
      Syncing disks.
    \end{lstlisting}

  \item Ahora deberá crear una nueva partición que albergue la información encriptada:

    \begin{lstlisting}[gobble=6,language=bash,style=bashinteract,escapechar=|]
      |\promptr| @@fdisk /dev/sdc@@
      Command (m for help): @@n@@
      Command action
      e   extended
      p   primary partition (1-4)
      @@p@@
      Partition number (1-4, default 1):
      Using default value 1
      First sector (2048-62530623, default 2048):
      Using default value 2048
      Last sector, +sectors or +size{K,M,G} (2048-62530623, default 62530623):
      Using default value 62530623

      Command (m for help): @@w@@
      The partition table has been altered!

      Calling ioctl() to re-read partition table.

      WARNING: Re-reading the partition table failed with error 16: Device or resource busy.
      The kernel still uses the old table. The new table will be used at
      the next reboot or after you run partprobe(8) or kpartx(8)
      Syncing disks.
    \end{lstlisting}

  \item La partición es ahora accesible en Linux en \path{/dev/sdc1}. Tras esto, formateo la partición usando la
    herramienta \lstinline!luksFormat! de Cryptsetup, creando una partición encriptada con LUKS con una palabra
    de paso. Antes de esto, no obstante, debo desmontar la unidad, sacar el pendrive y volver a introducirlo en
    el puerto USB, pues, si no hago esto, no me dejará formatearlo (pondrá
    \lstinline+Cannot format device /dev/sdc1 wich is still in use.+):

    \begin{lstlisting}[gobble=6,language=bash,style=bashinteract,escapechar=|]
      |\promptr| @@cryptsetup luksFormat /dev/sdc1@@

      WARNING!
      ========
      This will overwrite data on /dev/sdd1 irrevocably.

      Are you sure? (Type uppercase yes): @@YES@@
      Enter LUKS passphrase:
      Verify passphrase:
    \end{lstlisting}

  \item Ahora, necesito formatear la partición desencriptada. La elección del sistema de archivos dependerá del
    uso que le vaya a dar al pendrive: si quiero usarlo tanto en Linux como en Windows, necesitaré seleccionar
    FAT32 (\lstinline!vfat!); si se va a usar solo para Linux, puedo usar otro. Para formatear la partición
    desencriptada, necesito usar el comando \lstinline!luksOpen! de Cryptsetup y mapear un dispositivo, que yo
    llamo \path{LUKS001} (el mapeado se hace también con el comando luksOpen). El dispositivo mapeado estará
    presente en \path{/dev/mapper/LUKS001}. Tras esto, podemos formatear el dispositivo mapeado (con nombre
    \path{LUKS001}) y cerrarlo:

    \begin{lstlisting}[gobble=6,language=bash,style=bashinteract,escapechar=!]
      !\promptr! @@cryptsetup luksOpen /dev/sdc1 LUKS001@@
      Enter passphrase for /dev/sdc1:
      !\promptr! @@mkfs.vfat /dev/mapper/LUKS001 -n LUKS001@@
      mkfs.fat 3.0.26 (2014-03-07)
      unable to get drive geometry, using default 255/63
      !\promptr! @@cryptsetup luksClose LUKS001@@
    \end{lstlisting}

  \item Ahora podemos desconectar el pendrive y volverlo a conectar. Debería aparecer automáticamente un cuadro
    de diálogo cuando conectamos el pendrive pidiéndonos la palabra de paso. El autor del artículo en el que me
    baso, dice que ha comprobado que funciona con los entornos de escritorio Gnome y XFCE. Yo lo he hecho con
    Ubuntu 14.04, que usa Unity, que se basa en Gnome.

    A la hora de copiar ficheros de gran tamaño en la unidad encriptada, notará que tarda bastante. Es normal,
    al estar encriptada.

\end{enumerate}
