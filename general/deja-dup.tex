\section{Hacer copias de seguridad con Déjà Dup}\label{sec:deja-dup}
% --------------------------------------------------------------------------------------------------------------
Déjà Dup es el software que usaba en Linux para hacer mis copias de seguridad. Ahora mismo uso Time Machine, en
Mac OS X, que es el sistema operativo que uso ahora en mi sistema principal, para hacer copias de seguridad de
mi sistema. Mi intención es montarme un servidor casero de copias de seguridad haciendo uso de un Raspberry Pi 2
y el comando \lstinline!rsync! de Linux.

Déjà Dup es un front-end de Duplicity; ambos vienen de serie al instalar Ubuntu de escritorio (al menos, en las
versiones 12.04 y 14.04, que son las que yo he usado últimamente). Lo tengo configurado para que haga copias de
seguridad incrementales con frecuencia diaria y se almacenen copias como mínimo de los últimos 6 meses, siempre
y cuando lo permita el espacio disponible en la unidad de almacenamiento que uso, que es un pendrive de
\SI{32}{\gibi\byte} de marca Verbatim. Las copias se hacen en el directorio \path{/backups} en dicho pendrive;
es decir, el directorio, una vez automontada la unidad, es \path{/media/zfur/backups}. También hago copias en
DVD+Rs externos con la grabadora de discos ópticos que tengo para tener copias en ubicaciones y formatos
distintos. Los directorios que tengo seleccionados para hacerles copias de seguridad se muestran en la tabla
\ref{table:deja-dup}.

\begin{center}
  \begin{table}
    \taburowcolors[2] 2{color-tabu-a .. color-tabu-b}
%   \everyrow{\hline}
    \begin{tabu*}{l X[l]}
      \rowfont[c]{\bfseries\sffamily}
      Directorio                & Descripción\\
      \path{~/documents}        & Aquí tengo apuntes que no deseo que estén en el directorio de Dropbox.\\
      \path{~/Dropbox}          & Es mi directorio principal para todos mis documentos. Está sincronizada,
                                  obviamente, con la nube de Dropbox.\\
      \path{~/.passwordstore}   & Son las entradas que uso en el software PASS para gestionar mis claves de
                                  acceso.\\
      \path{~/docs-other}       & Aquí guardo, principalmente, libros y apuntes.\\
    \end{tabu*}\caption{Directorios a guardar en Déjà Dup}\label{table:deja-dup}
  \end{table}
\end{center}
