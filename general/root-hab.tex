\section{Habilitar y deshabilitar la cuenta de root}\label{sec:root-hab}
% --------------------------------------------------------------------------------------------------------------
Para habilitar la cuenta de root, debe introducir:

\begin{lstlisting}[gobble=2,language=bash,style=bashinteract,escapechar=!]
  !\promptu! @@sudo passwd root@@
\end{lstlisting}

\noindent Se le pedirá, entonces, dos veces, que introduzca la contraseña que desea usar para el usuario root.
El usuario con el que invoca el comando debe tener permisos de administración mediante \lstinline!sudo!. Se le
pedirá, entonces, antes, que introduzca la contraseña de su usuario actual para que se pueda ejecutar el comando
que ha introducido (a menos que haya introducido hace poco otro comando con \lstinline+sudo+ delante).

Para deshabilitarla, sólo deberá introducir:

\begin{lstlisting}[gobble=2,language=bash,style=bashinteract,escapechar=!]
  !\promptu! @@sudo passwd -dl root@@
\end{lstlisting}

\noindent Podría haberlo introducido desde su sesión de root y no pasaría nada: una vez que saliera de esa
sesión de shell, se eliminaría la cuenta de root. No olvide eliminar los ficheros o directorios que haya creado
para root, para no tener ocupando espacio en su disco duro ficheros que ya no usa.
