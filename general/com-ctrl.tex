\section{Comandos de control}\label{sec:com-ctrl}
% --------------------------------------------------------------------------------------------------------------
Existe una serie de comandos para el control de las tareas (\foreignlanguage{english}{\textit{jobs}}) de
Linux. En la tabla \ref{table:com-ctrl} hay una lista de los más importantes.

Si tiene un proceso que se ha abierto mediante la ejecución de algún comando de shell y desea pasarlo ahora a
segundo plano, puede hacerlo con la combinación de teclas \tecla{ctrl} + \tecla{z}. El shell le indicará que el
proceso se encuentra detenido. Con

\begin{lstlisting}[gobble=2,language=bash,style=bashinteract,escapechar=!]
  !\promptu! @@!\val{comando}! &@@
\end{lstlisting}

\noindent se abriría directamente en segundo plano el proceso que invoca el comando \val{comando}.

Si lo desea, puede hacer que se ejecute en segundo plano cualquier proceso, no sólo los que haya arrancado
mediante un comando de shell. Para esto, deberá buscar el número de proceso \val{núm} que tiene asignado
actualmente dicho proceso en su sistema operativo (por ejemplo, con el comando \lstinline!top! o
\lstinline!htop!) e introducir:

\begin{lstlisting}[gobble=2,language=bash,style=bashinteract,escapechar=!]
  !\promptu! @@bg %!\val{núm}!@@
\end{lstlisting}

\begin{center}
  \begin{table}
    \taburowcolors[2] 2{color-tabu-a .. color-tabu-b}
    \begin{tabu*}{l X[l]}
      \rowfont[c]{\bfseries\sffamily}
      Comando                           & Descripción\\
      \lstinline+jobs+                  & Muestra las tareas actuales.\\
      \lstinline+fg+                    & Reanuda la próxima tarea de la lista.\\
      \lstinline+fg \%<núm>+            & Reanuda la tarea número \val{núm}.\\
      \lstinline+bg+                    & Pone en segundo plano (\foreignlanguage{english}{\textit{background}})
                                          la siguiente tarea de la lista.\\
      \lstinline+bg \%<núm>+            & Pone en segundo plano la tarea número \val{núm}.\\
      \lstinline+kill \%<núm>+          & Mata la tarea con número \val{núm}.\\
      \lstinline+kill <señal> \%<núm>+  & Envía la señal \val{señal} a la tarea con número \val{núm}.\\
      \lstinline+disown \%<núm>+        & Elimina la pertenencia (\foreignlanguage{english}{\textit{disown}})
                                          del proceso (el terminal ya no será más el dueño) de modo que el
                                          comando vivirá incluso tras cerrar el terminal.\\
    \end{tabu*}\caption{Comandos de control más útiles en Linux}\label{table:com-ctrl}
  \end{table}
\end{center}
