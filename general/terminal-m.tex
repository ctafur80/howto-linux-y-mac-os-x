\section{Emulador de terminal en MAC OS X}\label{sec:terminal-m}
% -------------------------------------------------------------------------------------------------------------
El terminal de Mac OS X, de forma predeterminada está configurado para usar una paleta de sólo 16 colores. Si
desea poder usar una de 256, deberá hacer varias cosas. Hay quien cree que no le compensa hacer lo que explico
en este tutorial y prefiere simplemente instalar otro emulador de terminal que pueda usar 256 colores sin tener
que cambiar nada. Lo que a continuación explico es una copia de lo que puede encontrar en
\url{https://gist.github.com/shawnbot/3277580}.

Lo primero que debe hacer es asegurarse de que su emulador de terminal, es decir, el programa Terminal, está
declarado como \lstinline!xterm-256color!. Para esto, en la versión de Mac OS X El Capitán, deberá:

\begin{enumerate}
  \item pinchar en el menú Terminal,
  \item en el desplegable pinche en Preferences\ldots,
  \item seleccione en la columna de la izquierda el perfil que desea configurar,
  \item pinche luego en la pestaña Advanced y
  \item seleccione, en el menú desplegable junto a ``Declare terminal as:'', xterm-256color.
\end{enumerate}

Ahora deberá construir una versión de screen que soporte 256 colores, pues la que tiene ahora mismo no lo
soporta. Aquí se supondrá que tiene instalado y configurado el gestor de paquetes Homebrew:

\begin{lstlisting}[gobble=2,language=bash,style=bashinteract,escapechar=!]
  !\promptu! @@brew tap homebrew/dupes@@
  !\promptu! @@brew install homebrew/dupes/screen@@
\end{lstlisting}

Tras esto, usted tendrá el binario \path{/usr/local/bin/screen}.

Ahora, deberá, si no lo ha hecho anteriormente al configurar algún software, añadir, de forma permanente, la
ruta al comienzo de su variable de entorno \lstinline!$PATH!:

\begin{lstlisting}[gobble=2,language=bash,style=bashinteract,escapechar=!]
  !\promptu! @@echo 'export PATH=/usr/local/bin:$PATH' >> \ @@
  > @@/Users/!\val{user}!/.bashrc@@
\end{lstlisting}

Creo que lo normal es que este último paso lo realice justo tras instalar el gestor de paquetes Homebrew en su
sistema, pues los paquetes que se instalan mediante Homebrew se instalan bajo \path{/usr/local/}. Entonces, con
esta \lstinline!$PATH!, cualquier comando que escriba en su shell tratará de buscar el binario primero en
\path{/usr/local/}. Muchas veces puede desear instalar mediante Homebrew un programa que ya tiene porque desea
una versión más actualizada.
