\subsection{Bash en Mac OS X}\label{subsec:bash-m}
% --------------------------------------------------------------------------------------------------------------
Una vez que haya instalado Brew, puede instalar si lo desea el shell Bash por medio de Brew. Si hace esto
tendrá entonces dos versiones de dicho programa. Algo que no permite la versión antigua es hacer
autocompletion de Pass (Passwordstore). Para instalarlo mediante Brew, introduzca:

\begin{lstlisting}[gobble=2,language=bash,style=bashinteract,escapechar=!]
  !\promptu! @@brew install bash@@
\end{lstlisting}

\noindent También, vuelvo a instalar el paquete \lstinline!bash-completion!, aunque quizás ya lo tenga:

\begin{lstlisting}[gobble=2,language=bash,style=bashinteract,escapechar=!]
  !\promptu! @@brew install bash-completion@@
\end{lstlisting}

\noindent Ahora, debería configurar los programas de Mac OS para que tengan como ruta del shell la de la versión
nueva.  Es decir, en lugar de tener \path{/bin/bash}, deberán usar \path{/usr/local/bin/bash}. Por ejemplo, en
la aplicación Terminal, en el menú Terminal, en Preferences, en la pestaña General, en Shells open with: -->
Command (complete path): deberá introducir \path{/usr/local/bin/bash}.

\noindent También debería abrir una ventana de terminal e introducir:

\begin{lstlisting}[gobble=2,language=bash,style=bashinteract,escapechar=!]
  !\promptu! @@chsh -s /usr/local/bin/bash@@
\end{lstlisting}

Si desea que este nuevo Bash sea también el shell para root, tendrá que introducir este mismo comando
anteponiéndole sudo.

Otra cosa que debe hacer es ir a System Preferences, pinchar en Users \& Groups, Pinchar sobre el candado para
desbloquear la ventana de configuración y poder realizar cambios. Se le pedirá, para esto, que introduzca su
contraseña de usuario administrador. Pinche con el botón derecho sobre su usuario y aparecerá un menú en el
que pone Advanced Options\ldots Ahora aparecerá una nueva ventana. En el menú desplegable junto a Login shell:
introduzca la ruta absoluta de su nuevo shell; es decir, \path{/usr/local/bin/bash} y pinche en OK.
