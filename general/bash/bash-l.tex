\subsection{Bash en Linux}\label{subsec:bash-l}
% --------------------------------------------------------------------------------------------------------------
La configuración de Bash puede hacerla para un único usuario o para todo el sitema. Las opciones que desea para
un usuario, deberá introducirlas en \path{~/.bashrc}. Para todo el sistema
(\foreignlanguage{english}{\textit{systemwide}}), la ruta de configuración será \path{/etc/bash.bashrc} (en Mac
OS X es \path{/etc/bashrc}). En principio se aplican ambas configuraciones para su sistema, pero, en caso de
contradicción, prevalece para su usuario la de \path{~/.bashrc}. Es decir, se aplican en orden, primero la de
todo el sistema y después la del usuario sin restricciones.

Por ejemplo, si desea que el usuario \val{nom\_usuario} use los key bindings de Vi, en lugar de los de Emacs
(que son los que normalmente vienen de forma predeterminada en Linux), hay que añadir al arhivo
\path{~/.bashrc} lo siguiente:

\begin{lstlisting}[gobble=2,language=bash,style=bashinteract,escapechar=!]
  set -o vi
\end{lstlisting}

\noindent Con esto, estará en modo insert al comienzo, pero luego, si lo desea, puede pasar a modo comando,
pulsando la tecla \tecla{esc}. A dicho archivo podría añadir muchas otras opciones de configuración; por
ejemplo:

\begin{lstlisting}[gobble=2,language=bash,style=bashinteract,escapechar=!]
  TERM=screen-256color
\end{lstlisting}

\noindent para usar 256 colores en el emulador de terminal (Gnome Terminal, Konsole, etc.).

Yo, ahora prefiero que las configuraciones se hagan sólo para cada usuario. En mis cuentas en los distintos
sistemas que uso, suelo usar la misma configuración, que simplemente tengo que clonarla de mi repositorio de
Git.

En Mac OS X existe una peculiaridad. el fichero de configuración, en principio, sería \path{~/.bash_profile}.
Así que, lo que hago, es que en dicho fichero sólo dejo:

\begin{lstlisting}[gobble=2,language=bash,style=bashinteract,escapechar=!]
  source ~/.bashrc
\end{lstlisting}

\noindent y ya puedo usar \path{~/.bash_profile}, como en Linux. En
\url{http://apple.stackexchange.com/questions/12993/why-doesnt-bashrc-run-automatically} puede encontraar una
buena explicación---del usuario koiyu---de cómo difiere en Mac OS X respecto a Linux.
