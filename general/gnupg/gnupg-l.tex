\subsection{GnuPG en Linux}\label{subsec:gnupg-l}
% --------------------------------------------------------------------------------------------------------------
En Ubuntu, hago copias de seguridad incrementales diariamente de este directorio mediante el software Déjà Dup
(vea \nameref{sec:deja-dup}). En Fedora, no uso nada parecido, de momento. Tampoco es problema, ya que dicho
sistema operativo lo uso en una máquina virtual en un Mac, del que sí hago backups. En Fedora, al menos, no debe
preocuparse por instalar GnuPG, pues se instala con la instalación normal del sistema operativo.

Un manual muy bueno de cómo usar GnuPG se puede encontrar en
\url{https://www.digitalocean.com/community/tutorials/how-to-use-gpg-to-encrypt-and-sign-messages-on-an-ubuntu-12-04-vps}.

Para copiar un par de claves privada-pública a otro computador con Linux, deberá hacer lo que se explica en la
web \url{https://www.phildev.net/pgp/gpg_moving_keys.html}. Una vez que haya importado sus claves al nuevo
computador, debrá configurar Git y Pass si desea usarlos para almacenar de forma remota sus claves (vea
\nameref{sec:git} y \nameref{sec:passwordstore}).
