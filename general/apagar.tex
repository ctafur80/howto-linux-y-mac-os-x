\section{Apagar, suspender y reiniciar}\label{sec:apagar}
% -------------------------------------------------------------------------------------------------------------
Para apagar su sistema Linux desde el shell, tradicionalmente la mejor forma ha sido con:

\begin{lstlisting}[gobble=2,language=bash,style=bashinteract,escapechar=!]
  !\promptu! @@shutdown -h now@@
\end{lstlisting}

\noindent Con la opción \lstinline+-r+ (en lugar de \lstinline+-h+) se reiniciaría. No obstante, yo prefiero los
comandos de apagado \lstinline+poweroff+ (apagado), \lstinline+reboot+ (reinicio) y \lstinline+halt+
(suspender). Ahora que estas órdenes se ejecutan a tranvés de Systemd, los comandos serán:

\begin{lstlisting}[gobble=2,language=bash,style=bashinteract,escapechar=!]
  !\promptu! @@systemctl poweroff@@
  !\promptu! @@systemctl reboot@@
  !\promptu! @@systemctl halt@@
\end{lstlisting}

Debe tener ciertos privilegios para poder apagar el sistema. Lo normal es que exista en su sistema un grupo de
usuarios, llamado \texttt{power}, con la intención de dar permisos de apagado, suspensión, etc. Una solución
para dar dichos permisos puede ser editando el fichero \path{/etc/sudoers} (la mejor forma de editarlo es con el
comando \lstinline+visudo+, pues así evita que varias personas lo puedan modificar simultáneamente):

\begin{lstlisting}[gobble=2,language=bash,style=bashinteract,escapechar=!]
  %power ALL=(ALL) NOPASSWD: /usr/bin/systemctl -i poweroff,/usr/bin/systemctl -i halt,/usr/bin/systemctl -i reboot
\end{lstlisting}

\noindent \lstinline+NOPASSWD+ hace que no se le pida su contraseña al usuario al introducir alguno de estos
tres comandos.

Otra forma, mejor, en mi opinión, de dar este tipo de permisos a los usuarios del grupo \texttt{power} es con el
uso de Polkit (vea \nameref{subsec:polkit}).
