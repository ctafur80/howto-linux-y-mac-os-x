\section{Paquetes que suelo usar}\label{sec:tex-package}
% --------------------------------------------------------------------------------------------------------------
\LaTeX{} es en realidad un paquete de macros de \TeX. No es el único, hay otros como ConTeXt, más modernos. Aun
así, sigo prefiriendo usar \LaTeX{} porque es el que más gente usa y en el que más ayuda puedo encontrar a la
hora de buscar soluciones. No obstante, he de reconocer que ConTeXt es bastante más amigable. Se supone que la
versión 3 de \LaTeX{} será mucho más amigable y coherente y no habrá que estar usando paquetes para casi todo;
otro asunto es cuándo saldrá.

Los paquetes que suelo usar en \LaTeX{} tan a menudo que han reemplazado a los originales de la versión
\LaTeX{}2e son:

\begin{desc-package}

  \item[tabu] Este paquete permite hacer muchas cosas en relación a la creación de tablas que con los paquetes
    antiguos como \lstinline!tabular!, por ejemplo, eran muy difíciles. Permite crear, además de las columnas
    tradicionales (\lstinline+l+, \lstinline+c+, \lstinline+p+, etc.), columnas \lstinline+X+, que son
    elásticas. Así, se pueden tener columnas fijas y columnas elásticas dentro de una misma tabla. Si desea usar
    el comando \lstinline+\lstinline+ dentro de una tabla, deberá entonces usar \lstinline+tabu*+, en lugar de
    \lstinline+tabu+.

  \item[colortbl] Está relacionado con \lstinline+tabu+. Permite definir el color de los fondos de las celdas de
    una tabla.

  \item[siunitx] Este paquete es muy conveniente para la notación de las unidades en ciencia y tecnología, así
    como permite gran versatilidad en el uso de los numerales.

  \item[hyperref] Permite controlar los hiperenlaces y cómo se resaltan.

  \item[xcolor] Permite el empleo de muchos más colores de los que trae de fábrica \LaTeX{}2e.

  \item[graphicx] Permite la inclusión de gráficos en formato PDF, PNG o JPEG, por ejemplo.

  \item[verbatim] Para introducir código de forma muy directa sin preocuparse de introducir escapes.

  \item[fancyvrb] Para poder usar entornos verbatim pero con la posibilidad de introdicir comandos para que, por
    ejemplo, parte del texto esté en negrita, o en colores, o para que se pueda usar gobble.

  \item[titlesec] Para modificar la apariencia de los encabezados de capítulos, secciones, etc.

  \item[url] Permite usar el comando \lstinline+\url+, para introducción de URLs, o \lstinline+\path+, para
    introducir rutas de ficheros.

  \item[nameref] Con este paquete puedo usar referencias al nombre de una sección, tabla, figura, etc., y no
    sólo a su número.

  \item[inputenc] Codificación del fichero de entrada. Suelo usar siempre el estándar UTF-8 de Unicode.

  \item[fontenc] Codificación de fuentes.

  \item[titling]

  \item[amsmath] Paquete de la American Mathematical Society (AMS) para mejoras en la tipografía de notación
    matemática. Es bastante conveniente.

  \item[python] Sirve para usar código Python en documentos de \LaTeX{}. Yo lo uso para hacer gráficos con las
    librerías Matplotlib.

  \item[enumitem] Sirve para crear nuevos tipos de listas a partir de las predefinidas de \LaTeX{}, es decir, a
    partir de \lstinline!itemize!, \lstinline!enumerate! y \lstinline!description!.

  \item[dtklogos] Permite insertar los logos relacionados con el mundo \TeX{}, como, por ejemplo, ConTeXt,
    XeTeX, MiKTeX, etc. Cuando instalé la versión de 2015 de \TeX{} Live, no fui capaz de usar más este paquete.

  \item[imakeidx] Para crear un índice de palabras clave. Es más moderno que el método básico de \LaTeX{}2e.

  \item[glossaries] Para hacer glosarios.

  \item[floatrow] Para poder poner figuras en paralelo para que no ocupen tanto espacio.

  \item[tikz] Para hacer gráficos. Es una monstruosidad este paquete. El manual tiene muchas páginas.

  \item[circuitikz] Para representar esquemas de cirtuitos eléctricos y electrónicos, ya sean de teoría básica
    de circuitos, de electrónica analógica o de electrónica digital, etc. Este paquete emplea los gráficos del
    paquete \lstinline!tikz!, así que se podría considerar un subpaquete de aquél, pero hay que llamarlo con
    \lstinline+\usepackage{}+ como a cualquier otro paquete.

  \item[listings] Para hacer listados de código de programación. Dentro de los paquetes para esta función, creo
    que es el mejor. Aun así, no me parece muy bueno. Da problemas con los esquemas de colores de Vi y
    derivados; por ejemplo, considera el signo \$ como que abre/cierra el entorno display math. Puede modificar
    su esquema de colores o usar el de alguien; por ejemplo, vea
    \url{http://stackoverflow.com/questions/6738902/vim-syntax-highlighting-with-and-lstlistings-lstinline}.

\end{desc-package}
