\section{Instalación de un paquete}\label{sec:tex-install-p}
% --------------------------------------------------------------------------------------------------------------
El directorio donde se guardan normalmente los paquetes de \LaTeX{} es
\path{/usr/local/texlive/<a\~no>/texmf-dist/tex/generic}, pero no será ésa la ruta a la que debe añadir los
paquetes nuevos que cree, sino a \path{/usr/share/texmf/tex/latex}. Tendrá que crear un directorio que se llame
como el paquete que desea instalar. Luego, copie el fichero \val{paquete}.sty ahí. Reinicie la caché de \LaTeX{}
con

\begin{lstlisting}[gobble=2,language=bash,style=bashinteract,escapechar=!]
  !\promptu! @@texhash@@
\end{lstlisting}

\noindent o

\begin{lstlisting}[gobble=2,language=bash,style=bashinteract,escapechar=!]
  !\promptu! @@mktexlsr@@
\end{lstlisting}

Aun así, si ha realizado la instalación completa de \TeX\ Live o MacTeX, debería tener todos los paquetes. Yo no
me he visto en la necesidad de tener que instalar paquetes.
