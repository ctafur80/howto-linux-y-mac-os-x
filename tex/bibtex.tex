\section{BibTeX}\label{sec:tex-bibtex}
% --------------------------------------------------------------------------------------------------------------
Tengo una base de datos de BibTeX con los libros que puedo usar como referencia. Esta base de datos irá
creciendo continuamente. Estoy pensando en crear también una base de datos de webs de interés. La base de datos
de libros está en el fichero \path{/home/zfur/Dropbox/Documentos/latex/bib1.bib}. Las entradas las escribo con
minúscula, que también se puede, aunque mucha gente siga la costumbre antigua de poner en mayúsculas los nombres
de los campos. Una cosa que hago como configuración para todos mis documentos de \TeX{} es que dichas
referencias bibliográficas usen el estilo \lstinline+alpha+: \lstinline+\bibliographystyle{alpha}+; al ser para
todos los documentos, tuve que configurarlo en \path{/home/zfur/Dropbox/Documentos/latex/pre-base.tex}. A la
hora de hacer uso de estas referencias en un documento de \LaTeX{}, he intentado seguir las explicaciones de
\cite{latex-kopka}, pero he terminado haciéndolo de otro modo que me gusta más; aquí explico cómo.

Lo primero es introducir una sección de referencias, con el comando \lstinline+\bibliography+, en el documento
de \LaTeX{}. En mi caso,

\begin{lstlisting}[gobble=2,language=bash,style=bashinteract,escapechar=!]
  \bibliography{/home/zfur/Dropbox/Documentos/latex/bib1}
\end{lstlisting}

\noindent para el fichero \path{bib1.bib}, en esa ruta absoluta que se indica. Advierta que no he puesto la
extensión de nombre de fichero (\path{.bib}). Si se pone, luego, a la hora de enlazar, buscará en el fichero
\path{bib1.bib.bib} (está hecho así, bastante mal). En \path{bib1.bib} se encuentra la base de datos de BibTeX
en la que guardo los libros, artículos, etc. a los que puedo hacer referencia. Normalmente, prefiero que la
sección de referencias sea un apéndice, por lo que deberá estar tras el comando \lstinline+\appendix+. No tiene
por qué ser la primera sección de apéndice, así que no tiene por qué ir inmediatamente después de dicho comando.

Luego, para insertar una referencia, dentro de un documento, a la lista de referencias, deberá usar el comando
\lstinline!\cite{<clave>}!, \lstinline!\citep{<clave>}! o \lstinline!\citet{<clave>}!, donde \val{clave} es la
cadena usada como referencia en la entrada en la base de datos.

Finalmente, para crear el documento de \LaTeX{}, deberá hacer ahora otra cosa. Deberá primero ejecutar el
programa de paginación como hace normalmente: \lstinline+pdflatex+, \lstinline+xelatex+, etc.:

\begin{lstlisting}[gobble=2,language=bash,style=bashinteract,escapechar=!]
  !\promptu! @@pdflatex howto-linux.tex@@
\end{lstlisting}

\noindent Luego, deberá ejecutar, sobre el programa \lstinline+bibtex+, el fichero \path{*.aux} creado:

\begin{lstlisting}[gobble=2,language=bash,style=bashinteract,escapechar=!]
  !\promptu! @@bibtex howto-linux.aux@@
\end{lstlisting}

\noindent dos veces. Esto sólo se hará cuando modifique la base de datos, es decir, el fichero \path{*.bib}. Y,
finalmente, ejecutar otra vez el programa de paginación. Si no ha modificado la base de datos, podrá crear el
documento de \LaTeX{} como siempre lo ha hecho: \lstinline+pdflatex+, \lstinline+xetex+, \lstinline+context+,
etc.
