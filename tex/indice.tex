\section{Crear un índice de palabras clave}\label{sec:tex-indice}
% --------------------------------------------------------------------------------------------------------------
Para crear un índice de palabras clave en algún artículo o libro hecho con \LaTeX{}, suelo usar el paquete
\lstinline+imakeidx+, pues creo que es mejor que el método básico que viene con \LaTeX{}2e. Por tanto, en el
fichero \path{pre-base.tex} donde declaro los paquetes que uso para todo tipo de documentos, he introducido:

\begin{lstlisting}[gobble=2,style=bashinteract,escapechar=!]
  \usepackage{imakeidx}
\end{lstlisting}

\noindent Luego, en el propio documento, antes de

\begin{lstlisting}[gobble=2,style=bashinteract,escapechar=!]
  \begin{document}
\end{lstlisting}

\noindent introduciré

\begin{lstlisting}[gobble=2,style=bashinteract,escapechar=!]
  \makeindex[name=indice,title=!\val{título}!,columns=2]
\end{lstlisting}

\noindent donde \val{título} será el título que desea que aparezca en el encabezado del índice. Debe tener
cuidado pues los acentos quizás no los acepte si se introducen directamente, deberá usar en su lugar los
comandos para introducir acentos como, por ejemplo, \lstinline+\'a+ para tener \emph{á}. Puede crear varios
índices.  Yo suelo crear uno y por eso he puesto aquí un sólo índice, pero si deseara crear varios, debería
hacer un \lstinline+\makeindex+ por cada uno; éstos deberían tener distinto el valor del parámetro
\lstinline+name+.

También deberá introducir, si lo desea, algunos ajustes del índice con el comando

\begin{lstlisting}[gobble=2,language=bash,style=bashinteract,escapechar=!]
  \indexsetup{!\val{opciones}!}
\end{lstlisting}

\noindent donde estas opciones podrían ser, por ejemplo, \lstinline+noclearpage+, si desea que no se cree una
página nueva para el índice, es decir que éste vaya corrido. Este comando deberá ir también en el preámbulo del
documento.
