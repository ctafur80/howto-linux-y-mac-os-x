\section{Actualizar paquetes}\label{sec:update-p}
% --------------------------------------------------------------------------------------------------------------
En \url{http://tex.stackexchange.com/questions/55437/how-do-i-update-my-tex-distribution/55438#55438} puede ver
una explicación de cómo actualizar los paquetes de su distribución de \TeX{} Live.

Existe una aplicación de GUI de esta distribución, llamada \TeX{} Live Manager, que se ejecuta así:

\begin{lstlisting}[gobble=2,language=bash,style=bashinteract,escapechar=!]
  !\promptu! @@tlmgr --gui@@
\end{lstlisting}

\noindent Puede que le dé un error y necesite instalar el paquete \lstinline+perl-tk+, de apt-get, si es éste el
gestor de paquetes que usa en su distribución:

\begin{lstlisting}[gobble=2,language=bash,style=bashinteract,escapechar=!]
  !\promptr! @@apt-get install perl-tk --no-install-recommends@@
\end{lstlisting}

\noindent En DNF, el paquete es \lstinline+perl-Tk.x86_64+, para arquitectura x86 de 64 bits.

\TeX{} Live Manager tiene también la opción de invocarlo desde la línea de comandos para hacer la actualización
sin pasar por la interfaz gráfica. Deberá primero actualizar el propio \TeX{} Live Manager:

\begin{lstlisting}[gobble=2,language=bash,style=bashinteract,escapechar=!]
  !\promptu! @@tlmgr update --self@@
\end{lstlisting}

\noindent Y, luego, ya podrá actualizar los paquetes:

\begin{lstlisting}[gobble=2,language=bash,style=bashinteract,escapechar=!]
  !\promptu! @@tlmgr update --all@@
\end{lstlisting}

\noindent En Mac OS X necesito permisos de administrador para ejecutar estos comandos. Tengo que averiguar cómo
se hace la instalación para poder invocarlo con mi usuario sin usar \lstinline!sudo!. La ruta absoluta del
comando \lstinline!tlmgr! es \path{/usr/local/texlive/\val{a\~no}/bin/x86_64-linux/}, y se invoca porque
\path{/opt/texbin} está en la variable de entorno \lstinline!PATH! del usuario y \path{/opt/texbin} es un
vínculo simbólico a \path{/usr/local/texlive/\val{a\~no}/bin/x86_64-linux/}

Existe otro comando, mejor, para actualizar los paquetes:

\begin{lstlisting}[gobble=2,language=bash,style=bashinteract,escapechar=!]
  !\promptr! @@tlmgr update --self --all --reinstall-forcibly-removed@@
\end{lstlisting}

Creo que los paquetes se actualizan durante el año que dura su versión de distribución. Cuando salga una versión
nueva, deberá entonces actualizar la distribución completa para poder seguir teniendo actualizados los paquetes
(vea \nameref{sec:tex-upgrade}).
