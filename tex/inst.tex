\section{Instalar \TeX{} Live en Linux}\label{sec:inst-tex-l}
% --------------------------------------------------------------------------------------------------------------
Para elaborar este tutorial, me he basado, además de en la docuentación oficial de \TeX{} Live
(\url{http://www.tug.org/texlive/quickinstall.html}), en el comentario del usuario Silke, en este hilo
(\url{http://tex.stackexchange.com/questions/1092/how-to-install-vanilla-texlive-on-debian-or-ubuntu}) de los
foros de \TeX{} de Stack Exchange. Al final he optado por hacerlo de una forma algo personal. Se trata de una
instalación para el grupo de usuarios \texttt{tex}, que previamente a la instalación deberé haber creado.

La mejor forma de instalar \TeX{} Live es desde el script instalador con nombre \path{install-tl} que hay en su
web para descargar. Es mejor que con cualquier gestor de paquetes de Linux, tales como apt-get en Ubuntu, o DNF
en Fedora. Esto me permite tener más actualizados los paquetes que componen \TeX{} Live. Cada año por verano
sale una versión nueva de \TeX{} Live. La última es la de 2015, que salió el 24 de junio.

Lo primero es descargarse de \url{http://www.tug.org/texlive/acquire-netinstall.html} el archivo
\path{install-tl-unx.tar.gz}:

\begin{lstlisting}[gobble=2,language=bash,style=bashinteract,escapechar=!]
  !\promptu! @@wget http://mirror.ctan.org/systems/texlive/tlnet/install-tl-unx.tar.gz@@
\end{lstlisting}

\noindent Luego lo descomprimirá donde quiera (puede valer dentro de \path{~/Downloads/}, por ejemplo):

\begin{lstlisting}[gobble=2,language=bash,style=bashinteract,escapechar=!]
  !\promptu! @@tar zxvf install-tl.tar.gz@@
\end{lstlisting}

Antes de seguir con la instalación, debe limpiar las instalaciones existentes que tenga de \TeX Live:

\begin{lstlisting}[gobble=2,language=bash,style=bashinteract,escapechar=!]
  !\promptr! @@rm -rf /usr/local/texlive/!\val{año}!/@@
  !\promptr! @@rm -rf ~/.texlive!\val{año}!/@@
\end{lstlisting}

\noindent donde \val{año} es el año de la versión de \TeX{} Live que tenía instalada y desea eliminar; las
versiones de \TeX{} Live se sacan anualmente, normalmente en verano. Puede que tenga más de una versión
instalada. Lo mejor en dicho caso sería eliminarlas todas e instalar la última. Realmente no es necesario
eliminar las instalaciones anteriores para instalar otra; pueden coexistir varias a la vez. Cada una se
encontraría en su directorio \path{/usr/local/texlive/<a\~no>/}, donde ese \val{año} es donde se separan las
distintas instalaciones.

Ahora deberá preparar el directorio \path{/usr/local/texlive/}. Seguramente, si ya existe, tendrá como dueño a
root, será del grupo de root y los permisos Unix que tiene serán 775. Según mi método, deberá primero crear en
su sistema el grupo tex:

\begin{lstlisting}[gobble=2,language=bash,style=bashinteract,escapechar=!]
  !\promptr! @@groupadd tex@@
\end{lstlisting}

\noindent Ahora, cambiar el grupo del directorio:

\begin{lstlisting}[gobble=2,language=bash,style=bashinteract,escapechar=!]
  !\promptr! @@chown root:tex /usr/local/texlive/@@
\end{lstlisting}

\noindent Ahora, asignar permisos de escritura para dicho directorio a los miembros del grupo tex:

\begin{lstlisting}[gobble=2,language=bash,style=bashinteract,escapechar=!]
  !\promptr! @@chmod g+w /usr/local/texlive/@@
\end{lstlisting}

Una vez hecho esto, deberá añadir al/los usuario(s) que desee al grupo tex:

\begin{lstlisting}[gobble=2,language=bash,style=bashinteract,escapechar=!]
  !\promptr! @@usermod -aG tex !\val{n\_usuario}!@@
\end{lstlisting}

\noindent Cuidado, no olvide la opción \lstinline!a!; si lo hace, dicho usuario perderá todos los demás grupos a
los que pertenecía. Ahora puede comprobar que el usuario \val{n\_usuario} pertenece al grupo tex:

\begin{lstlisting}[gobble=2,language=bash,style=bashinteract,escapechar=!]
  !\promptr! @@groups !\val{n\_usuario}!@@
\end{lstlisting}

Ahora deberá ejecutar el instalador pero con el usuario \val{n\_usuario}. Para ello, entre en el directorio
donde haya desarchivado y descomprimido el archivo \path{install-tl-unx.tar.gz}, que ahora sería simplemente
\texttt{install-tl-\val{año}\val{mes}\val{día}} y, una vez dentro del mismo, ejecute desde el shell:

\begin{lstlisting}[gobble=2,language=bash,style=bashinteract,escapechar=!]
  !\promptu! @@./install-tl@@
\end{lstlisting}

\noindent Ahora aparecerá una pantalla interactiva sobre qué desea hacer. Para instalar basta con pulsar
\tecla{i} y luego \tecla{intro}. Como puede ver en la Quick Install, existen varias opciones a la hora de
instalar. Si lo hace como he explicado aquí, se le instalarán todos los paquetes de \TeX{} Live, cosa que
tarda un un rato, aunque es algo que se puede hacer en un terminal mientras se hacen otras cosas y no consume
casi nada de recursos ni ancho de banda. Se pueden cambiar algunas opciones antes de la instalación, pero las
predefinidas creo que bastarán a casi todo el mundo. Aun así, esas opciones pueden cambiarse dentro del
\path{*.tex} para cada documento en particular. Al hacer la instalación completa, además de \LaTeX,
PDF\LaTeX{} y Plain \TeX, también pueden usarse, XeTeX, LuaTeX y ConTeXt, aunque yo prefiero seguir usando
\LaTeX, pues en los otros no encuentro tantos paquetes.

Ahora hay que hacer algo importante, pues, si no se hace, no se podrán ejecutar los comandos de \LaTeX. Lo que
se debe hacer es agregar, de forma permanente, la dirección donde se encuentran los ejecutables de \LaTeX{} a
la variable de entorno \lstinline+PATH+ del usuario (o de los usuarios) que vaya(n) a usar \TeX{} Live. Existen
muchas formas de hacerlo; puede ver una explicación de las distintas formas en
\url{http://stackoverflow.com/questions/14637979/how-to-permanently-set-path-on-linux}. Primero deberá crear
un vínculo simbólico:

\begin{lstlisting}[gobble=2,language=bash,style=bashinteract,escapechar=!]
  !\promptr! @@ln -s /usr/local/texlive/!\val{año}!/bin/* /opt/texbin@@
\end{lstlisting}

Ahora deberá añadir la ruta \path{/opt/texbin} a su variable de entorno \lstinline+PATH+:

\begin{lstlisting}[gobble=2,language=bash,style=bashinteract,escapechar=!]
  !\promptu! @@echo 'export PATH=$PATH:/opt/texbin' >> /home/!\val{user}!/.bashrc@@
\end{lstlisting}

\noindent Ese \val{user} es porque mi directorio personal es \path{/home/<user>}. Ahora debe cerrar su sesión
y volver a entrar, para que los cambios tengan efecto; o, si lo prefiere, reinicie su sistema. Puede ver, tras
el reinicio/reentrada, cómo ha quedado la variable de entorno \lstinline+PATH+ con:

\begin{lstlisting}[gobble=2,language=bash,style=bashinteract,escapechar=!]
  !\promptu! @@echo $PATH@@
\end{lstlisting}

\noindent También puede comprobar que ha modificado correctamente la variable \lstinline!PATH!:

\begin{lstlisting}[gobble=2,language=bash,style=bashinteract,escapechar=!]
  !\promptu! @@which tex@@
  /opt/texbin/tex
\end{lstlisting}

\noindent Debería dar ese resultado que se muestra.

También es muy útil instalar las librerías Matplotlib, para poder representar gráficas 2D mediante la inserción
de código Python en el documento (vea \nameref{subsec:matplotlib}). En cuanto a las fuentes, la instalación
completa de \TeX{} Live instala muchas fuentes de gran calidad, como las Adobe Source, las \TeX{} Gyre y las
Linux Liberation (para más información, vea \nameref{sec:adobe-source} e \nameref{sec:lin-libertine}).
 
Para entornos Windows creo que es mejor instalar MiKTeX. En Mac OS X, MacTeX.
