\section{Crear un glosario}\label{sec:tex-glos}
% --------------------------------------------------------------------------------------------------------------
Normalmente, creo un glosario en un documento cuando el documento es bastante grande. Al ser el documento
grande, suelo hacer los capítulos o secciones (según se trate de formato \lstinline+book+ o \lstinline+article+,
respectivamente) como ficheros separados a los que luego llamo dentro del fichero principal con el comando
\lstinline+\input+.

Así, pues, suponga que desea incluir el preámbulo en un fichero llamado \path{preamb.tex} en el mismo directorio
que los demás que forman el documento de \LaTeX. En este fichero \path{preamb.tex} deberá incluir:

\begin{enumerate}

  \item La declaración del uso del paquete \lstinline+glossaries+, es decir,
    \lstinline+\usepackage{glossaries}+. Prefiero usar este paquete al modo predeterminado que viene con
    \LaTeX{}2e.

  \item El comando \lstinline+\makenoidxglossaries+.

  \item Las entradas del glosario, que tendrán la forma:

    \begin{lstlisting}[gobble=6,language=bash,style=bashinteract,escapechar=!]
      \newglossaryentry{!\val{entrada}!}{name={!\val{name}!},plural={!\val{plural}!},description={!\val{description}!}}
    \end{lstlisting}

    \noindent donde \val{entrada} se refiere a la referencia que uso para luego llamarlo con el comando
    \lstinline+\gls+, para hacer referencia a \val{name}, o \lstinline+\glspl+, para referenciar a \val{plural}. En
    \val{description} deberá introducir la descripción del concepto que desea explicar; no debe poner punto al
    final de la descripción, pues ya se encarga el comando de hacerlo por sí mismo. De estas entradas, se verán
    luego sólo las que se usen realmente como referencia en el texto, es decir, las que se hayan referenciado
    mediante el comando \lstinline+\gls+ o \lstinline+\glspl+.

\end{enumerate}

Luego, en el fichero principal que llama a los demás, se deberá introducir, donde se quiera incluir el glosario,
\lstinline+\printnoidxglossaries+.
