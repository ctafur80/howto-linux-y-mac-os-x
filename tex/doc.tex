\section{Documentación}\label{sec:tex-doc}
% --------------------------------------------------------------------------------------------------------------
Tengo varios documentos que he hecho para tener como referencia en varios campos. El \emph{Howto --- Linux}, por
ejemplo, es decir, este mismo documento, es un documento en el que explico asuntos sobre la instalación,
configuración y uso de cierto software que uso en Linux, aunque la mayoría existe también para otras
plataformas, como Microsoft Windows, por ejemplo. La mayoría de estos documentos están hechos con \LaTeX{} y
tienen bastante trabajo. Los preámbulos los guardo en el directorio
\path{/home/zfur/Dropbox/Documentos/importante/latex/}; también están en dicho directorio la sección general
sobre notación que uso en la mayoría de mis documentos (\path{notacion-artcl.tex}) y las bases de datos de
BibTeX. Estos ficheros los usaré como base, pero pueden ser complementados con ficheros de preámbulo o notación
en el mismo directorio que los demás que forman el documento en concreto.

Lo normal es usar el comando \verb+\input+ para introducir esos preámbulos o sección de notación. Por ejemplo,

\begin{lstlisting}[gobble=2,language=bash,style=bashinteract,escapechar=!]
  \input{/home/zfur/Dropbox/Documentos/latex/frmat-artc.tex}
  \input{/home/zfur/Dropbox/Documentos/latex/lang-sp.tex}
  \input{/home/zfur/Dropbox/Documentos/latex/pre-base.tex}
  \input{/home/zfur/Dropbox/Documentos/latex/pre-artcl.tex}
\end{lstlisting}

\noindent podría usarse para un artículo. Es necesario que se pongan en ese orden. También puedo poner después
un fichero de preámbulo del propio documento, pero sólo lo creo cuando modifico muchas cosas para ese documento;
lo normal es que sean muy pocas y, por tanto, las dejo en el fichero principal que llama a los demás.
