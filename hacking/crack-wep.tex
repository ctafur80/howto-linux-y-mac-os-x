\section{Cracking de redes wireless con seguridad WEP}\label{sec:crack-wep}
% -------------------------------------------------------------------------------------------------------------
Para hacer este tutorial me he basado en el que puede encontrar en
\url{http://www.aircrack-ng.org/doku.php?id=simple_wep_crack}, aunque hay cosas que he hecho de forma distinta;
tenga en cuenta que éste último es del año 2010.

Aquí se supone que usted tiene en su sistema una WNIC (tarjeta de interfaz wireless de red; en ingles,
\foreignlanguage{english}{\textit{wireless network interface card}}) con los controladores necesarios para hacer
inyección de paquetes. Tenga en cuenta que, según qué chip tenga en dicha tarjeta, se le puede hacer muy
complicado el proceso de instalar los controladores para inyección de paquetes, así que, si va a comprar una
para practicar estos tutoriales, lo mejor es que se informe antes de cuál le conviene comprar. Puede consultar
el siguiente enlace de enero de 2015:
\url{http://www.wirelesshack.org/top-kali-linux-compatible-usb-adapters-dongles-2015.html}. La mía es una
TP-Link TL-WN772N, que ha funcionado muy bien, en todas las distribuciones de Linux en las que la he probado,
sin necesidad de instalar ``manualmente'' los controladores.

Este ataque que se describe aquí se basa en usar el software Aireplay-ng para replay paquetes ARP para generar
nuevos IVs (vectores de inicialización, \foreignlanguage{english}{\textit{initialization vectors}}). Luego,
estos IVs los usará la herramienta de software Aircrack-ng para crackear la clave WEP.

Antes de nada, debe asegurarse de cumplir ciertos requisitos:

\begin{itemize}
  \item La WNIC que va a usar para este tutorial tiene controladores que le permitan realizar inyección
    de paquetes.
  \item Su WNIC se encuentra lo bastante cerca del AP para emitir y recibir paquetes. Advierta que, aunque pueda
    transmitir paquetes al AP (punto de acceso, se sobreentiende que wireless;
    \foreignlanguage{english}{\textit{(wireless) access point}}), quizás no pueda recibirlos, pues es fácil
    variar la potencia a la que transmite su WNIC, mientras que para mejorar su recepción lo único que puede
    hacer es cambiar la antena de su WNIC por una de más \si{\deci\bel i}.
  \item Existe al menos un cliente conectado, por cable o inalámbricamente (?????????TKTKTKTK), al AP que quiere
    atacar y este/estos cliente(s) está(n) activo(s). La razón está en que este tutorial presupone que usted
    pueda recibir al menos un paquete de solicitud de ARP.
\end{itemize}

Recuerde la notación empleada en todo este capítulo (vea \nameref{chapter:hacking}).

Para crackear una clave WEP de un AP, usted necesita recolectar muchos IVs. El tráfico normal de red no suele
generar muchos IVs en poco tiempo. Usted podría crackear un AP con seguridad WEP simplemente monitorizando los
paquetes wireless en la red y guardándolos para analizarlos, pero esto le llevaría seguramente mucho tiempo.
Así, pues, será mejor usar una técnica, llamada inyección, para acelerar el proceso. La inyección consiste, en
este ámbito, en hacer que el AP reenvíe paquetes seleccionados de forma rápida una y otra vez. Gracias a esto,
podrá capturar un gran número de IVs en poco tiempo.

\begin{enumerate}

  \item Pase \val{nic\_att} a modo monitor.

    El modo en el que suelen estar configuradas las WNIC es managed, que consiste en que su interfaz sólo
    ``escuche'' los paquetes que se envían a ella (incluidos los de broadcast). Las interfaces de la wireless
    LAN de los AP se encuentran en modo master. El modo monitor le permitirá ``escuchar'' todos los paquetes
    wifi que hay en las cercanías (en realidad, las señales electromagnéticas de dichos paquetes). Para inyectar
    paquetes, deberá estar en modo monitor (creo que también puede hacerlo en algún otro modo, como el modo
    promiscuo, pero no viene al caso).

    Para pasarla a modo monitor, antes deberá ponerla en down:

    \begin{lstlisting}[gobble=6,language=bash,style=bashinteract,escapechar=!]
      !\promptr! @@ifconfig !\val{nic\_att}! down@@
    \end{lstlisting}

    Ahora, con \lstinline!airmon-ng! debería pasar a modo monitor:

    \begin{lstlisting}[gobble=6,language=bash,style=bashinteract,escapechar=!]
      !\promptr! @@airmon-ng start !\val{nic\_att}!@@
    \end{lstlisting}

    Ahora podrá comprobar, con

    \begin{lstlisting}[gobble=6,language=bash,style=bashinteract,escapechar=!]
      !\promptu! @@iwconfig@@
    \end{lstlisting}

    \noindent que \val{nic\_att} ha pasado a modo monitor. Verá que el nombre de \val{nic\_att} ha cambiado;
    aquí, la nueva designación la representaremos de forma genérica como \val{nic\_att\_mon}, que en su caso
    será distinta.

    Ahora, si lo desea, puede hacer MAC spoofing de su interfaz mon (vea \nameref{sec:mac-spoofing}).

  \item Probar que su interfaz \val{nic\_att\_mon} recibe la señal del AP a atacar con la suficiente potencia
    para poder realizar la inyección de paquetes. Para esto, deberá introducir:

    \begin{lstlisting}[gobble=6,language=bash,style=bashinteract,escapechar=!]
      !\promptr! @@aireplay-ng -9 -e !\val{essid\_ap}!a !\val{bssid\_ap}! !\val{nic\_att\_mon}!@@
    \end{lstlisting}

    \noindent Ese \lstinline!9! indica test de inyección. El sistema debería responder algo parecido a lo
    siguiente:

    \begin{lstlisting}[gobble=6,style=bashinteract]
      09:23:35  Waiting for beacon frame (BSSID: 00:14:6C:7E:40:80) on channel 9
      09:23:35  Trying broadcast probe requests...
      09:23:35  Injection is working!
      09:23:37  Found 1 AP

      09:23:37  Trying directed probe requests...
      09:23:37  00:14:6C:7E:40:80 - channel: 9 - 'teddy'
      09:23:39  Ping (min/avg/max): 1.827ms/68.145ms/111.610ms Power: 33.73
      09:23:39  30/30: 100%
    \end{lstlisting}

    \noindent Debe fijarse en la última línea. El porcentaje que le muestra debe ser cercano (si no igual) al
    \SI{100}{\percent}. Si no es así, es que la señal que recibe no le llega con suficiente potencia,
    seguramente, porque se encuentra demasiado lejos del AP. Una forma de solucionarlo sería usar una WNIC con
    una antena con más \si{\deci\bel i}, tal y como se dijo antes. Si el valor es \SI{0}{\percent}, seguramente
    es síntoma de que su inyección no está funcionando; para solucionarlo puede parchear sus controladores o
    usar otros.

  \item Ejecute Airodump-ng para capturar los IVs.

    Es aconsejable que este comando lo introduzca en otra consola, pues deberá ejecutarse mientras hace otras
    cosas. Podría abrir una nueva ventana o pestaña de su emulador de terminal, una nueva ``pestaña'' de un
    multiplexor de terminal como Tmux (vea \nameref{sec:tmux}), o, si lo prefiere, ejecutarlo en segundo plano
    (\foreignlanguage{english}{\textit{background}}, vea \nameref{sec:com-ctrl}) y volver al mismo cuando lo
    necesite.

    \begin{lstlisting}[gobble=6,language=bash,style=bashinteract,escapechar=!]
      !\promptr! @@airodump-ng -c !\val{canal\_ap}! -b !\val{bssid\_ap}! \ @@
      > @@-w !\val{pref\_out}! !\val{nic\_att\_mon}!@@
    \end{lstlisting}

    \noindent donde \val{pref\_out} es el prefijo de los nombres de los ficheros que contendrán la salida del
    comando, es decir, que contendrán los IVs.

    Mientras está teniendo lugar la inyección (cosa que se hará posteriormente), deberá aparecerle, en este
    comando, algo así:

    \begin{lstlisting}[gobble=6,language=bash,style=bashinteract]
      CH  9 ][ Elapsed: 8 mins ][ 2007-03-21 19:25

      BSSID              PWR RXQ  Beacons    #Data, #/s  CH  MB  ENC  CIPHER AUTH ESSID

      00:14:6C:7E:40:80   42 100     5240   178307  338   9  54  WEP  WEP         teddy

      BSSID              STATION            PWR  Lost  Packets  Probes

      00:14:6C:7E:40:80  00:0F:B5:88:AC:82   42     0   183782
    \end{lstlisting}

  \item Use Aireplay-ng para hacer una autenticación falsa con el AP.

    Es necesario, para que el AP acepte un paquete de una interfaz, que dicha interfaz se encuentre ya asociada
    a dicho AP. En caso de que no sea así, el AP ignorará el paquete y enviará un paquete DeAuthenticarion en
    claro. Entonces, no se crearán nuevos IVs puesto que el AP estará ignorando todos los paquetes inyectados.
    La interfaz que usted use para inyección ha de estar asociada con el AP mediante el uso de un autentificador
    falso o bien usando una dirección MAC de un cliente ya asociado al AP. Aquí usaremos un autentificador
    falso:

    \begin{lstlisting}[gobble=6,language=bash,style=bashinteract,escapechar=!]
      !\promptr! @@aireplay-ng -1 0 -e !\val{essid\_ap}! -a !\val{bssid\_ap}! \ @@
      > @@-h !\val{nic\_att\_mon\_mac}! !\val{nic\_att\_mon}!@@
    \end{lstlisting}

    \noindent La opción \lstinline!-1! significa autenticación falsa y \lstinline!0! es el tiempo, en segundos,
    de reasociación. Si ha tenido éxito, el comando le devolverá algo parecido a lo siguiente:

    \begin{lstlisting}[gobble=6,language=bash,style=bashinteract]
      18:18:20  Sending Authentication Request
      18:18:20  Authentication successful
      18:18:20  Sending Association Request
      18:18:20  Association successful :-)
    \end{lstlisting}

  \item Ejecute Aireplay-ng en modo solicitud de replay de ARP.

    El propósito de este paso es ejecutar Aireplay-ng en un modo que escuche las solicitudes de ARP y
    posteriormente las reinyecte en la red. La razón por la que se usan paquetes de solicitud de ARP es porque
    el AP normalmente los rebroadcasteará (menuda palabra me acabo de inventar) y generará un nuevo IV. Así se
    podrán crear muchos IVs en poco tiempo.

    Abra otra ventana/pestaña, etc. e introduzca:

    \begin{lstlisting}[gobble=6,language=bash,style=bashinteract,escapechar=!]
      !\promptr! @@aireplay-ng -3 -b !\val{bssid\_ap}! \ @@
      > @@-h !\val{nic\_att\_mon\_mac}! !\val{nic\_att\_mon}!@@
    \end{lstlisting}

    \noindent Esto empezará a escuchar solicitudes de ARP y, en cuanto escuche la primera, Aireplay-ng comenzará
    inmediatamente a inyectarlas. Debería aparecerle algo así:

    \begin{lstlisting}[gobble=6,language=bash,style=bashinteract]
      Saving ARP requests in replay_arp-0321-191525.cap
      You should also start airodump-ng to capture replies.
      Read 629399 packets (got 316283 ARP requests), sent 210955 packets...
    \end{lstlisting}

    \noindent A mí, al principio, me ponía \lstinline!sent 0 paquets! y no cambiaba. Entonces fui al ordenador
    que tenía de cliente conectado al AP (porque no era mentira eso de que este tutorial es sólo para
    divertirse; está feo eso de meterse en la wifi de tu vecina anciana) e hice un ping a la interfaz wireless
    LAN del AP. Entonces el número cambió y siguió cambiando indefinidamente. Quizás es que hay que esperar a
    que algún cliente haga algo en la red.

    Ahora puede comprobar que en la shell en la que tiene ejecutando el comando \lstinline!airodump-ng!, bajo la
    columa \lstinline+#/s+ le aparece un número mayor de \num{100}. Si no es así, no está funcionando bien la
    cosa.

  \item Ejecute Aircrack-ng para obtener la clave WEP.

    Ahora, con el uso de la herramienta de software Aircrack-ng, podrá obtener, a partir de los IVs que ha
    recolectado de los pasos anteriores, la clave WEP del AP.

    Aquí mostraré dos métodos. El primero se llama PTW. Deberá introducir el comando siguiente en un shell
    nuevo:

    \begin{lstlisting}[gobble=6,language=bash,style=bashinteract,escapechar=!]
      !\promptr! @@aircrack-ng -b !\val{bssid\_ap}! !\val{pref-out}!*.cap@@
    \end{lstlisting}

    El otro método, llamado FMS/Korek, consiste en lo siguiente:

    \begin{lstlisting}[gobble=6,language=bash,style=bashinteract,escapechar=!]
      !\promptr! @@aircrack-ng -K -b !\val{bssid\_ap}! !\val{pref-out}!*.cap@@
    \end{lstlisting}

    A mí me tardó menos el primer método. Según dice el manual, para el ataque FMS/Korek, se necesitan unos
    \num{250000} IVs para crackear una clave WEP de \SI{64}{\bit}, y aproximadamente \num{150000} para una de
    \SI{128}{\bit}. Para el ataque PTW, serían unos \num{20000} para una de \SI{64}{\bit} y entre \num{40000} y
    \num{85000} para una de \SI{128}{\bit}.

    Algo curioso es que el método FMS/Korek me dio dos resultados. Quizás sean válidos ambos\ldots o no. Tengo
    que probarlo.

\end{enumerate}
