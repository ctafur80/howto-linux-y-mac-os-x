\section{MAC spoofing}\label{sec:mac-spoofing}
% -------------------------------------------------------------------------------------------------------------
El MAC spoofing podría traducirse como ``enmascaramiento de MAC''; aquí MAC hace alusión a \emph{dirección MAC}
(también conocida como ``dirección física'' o`` dirección hardware''). Consiste en cambiar la dirección MAC que
muestra nuestro equipo, normalmente antes de realizar un ataque wifi, con la intención de que no se nos pueda
identificar. Advierta que la dirección MAC es única para cada NIC en todo el mundo y una organización con el
suficiente poder, como, por ejemplo, una perteneciente a un estado, podría consultar los registros de ventas de
las principales empresas que venden ordenadores y NICs wireless y averiguar quién compró su ordenador o NIC;
sobretodo si pagó con tarjeta de crédito o débito dicha transacción comercial.

Para hacer MAC spoofing en su \val{nic\_att}, deberá hacer lo siguiente

\begin{lstlisting}[gobble=2,language=bash,style=bashinteract,escapechar=!]
  !\promptr! @@ifconfig !\val{nic\_att\_mon}! down@@
  !\promptr! @@macchanger -r !\val{nic\_att\_mon}!@@
  !\promptr! @@iwconfig !\val{nic\_att\_mon}! mode monitor@@
  !\promptr! @@ifconfig !\val{nic\_att\_mon}! up@@
\end{lstlisting}
