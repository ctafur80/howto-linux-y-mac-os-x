\section{Tareas post-instalación del sistema operativo}\label{sec:ubuntu-post-install}
% --------------------------------------------------------------------------------------------------------------
Después de instalar Linux, concretamente, Ubuntu, que es la distribución que suelo tener siempre, pues es la más
conocida y usada, debo hacer una serie de cosas:

\begin{itemize}
  \item Instalar \TeX{} Live (vea \nameref{sec:tex-inst} e \nameref{sec:matplotlib}).
  \item Instalar htop, que es como top pero más moderno. Sirve para ver en tiempo real el empleo de los recursos
    del sistema por los distintos procesos.
  \item Instalar tmux, que es un multiplexor de terminal. Está bastante bien por la comodidad que otorga. Había
    otro programa parecido, llamado Terminator, pero creo que tmux es mejor.
\end{itemize}

Si instala \TeX{} Live, se instalarán varias fuentes de gran calidad automáticamente en la jerarquía de
directorios de \TeX{} (por ejemplo, en \path{/usr/local/texlive/2013/texmf-dist/fonts/opentype/}), como las
Adobe Source, las \TeX{} Gyre o las Linux Liberation. Si no va a instalar \TeX{} Live y desea tener estas
fuentes, he creado en este mismo directorio varios scripts de Bash para instalar estas fuentes en la ruta usual
de Linux para las fuentes (\path{/usr/share/fonts/truetype}) para que puedan ser usadas por todas las cuentas de
usuario de su sistema (vea \nameref{sec:adobe-source} e \nameref{sec:lin-libertine}).

Algo que solía configurar para el arranque de Linux es que se abra un terminal de GNOME con varias pestañas. El
comando que añado es el siguiente:

\begin{lstlisting}[gobble=2,language=bash,style=bashinteract]
  gnome-terminal --tab-with-profile=uno --tab-with-profile=dos --working-directory=/home/ctafur/Dropbox/Documentos/otros --tab-with-profile=tres --working-directory=/home/ctafur/Dropbox/Documentos/importante --tab-with-profile=cuatro
\end{lstlisting}

\noindent Ojo, las rutas hay que ponerlas de forma absoluta; no se puede poner \lstinline+~+ en lugar de
\lstinline+/home/zfur/+. No obstante, ahora prefiero ir abriéndolas yo en los espacios de trabajo conforme las
necesito; se abren con la combinación de teclas \tecla{alt} \tecla{ctrl} + \tecla{t}.
