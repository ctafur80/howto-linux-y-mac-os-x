\section{Copiar y pegar texto}\label{sec:vim-cop-peg}
% --------------------------------------------------------------------------------------------------------------
Después de comerme mucho la cabeza, lo mejor creo que es copiar en el portapapeles de su entorno de escritorio
(el de Mac OS X, Windows 10, Gnome, KDE, etc.) y pegarlo en el propio documento con la combinación que tenga
para pegar en su emulador de terminal (por ejemplo, en Mac OS X, con \tecla{cmd} + \tecla{v}; en Linux,
\tecla{ctrl} + \tecla{shift} + \tecla{v}).  Pero para poder hacer esto debe antes, en Vim, habilitar la opción
de pegado, lo cual se realiza introduciendo en modo comando:

\begin{lstlisting}[gobble=2,style=bashinteract,escapechar=!]
  :set paste
\end{lstlisting}

\noindent y posteriormente pasando a modo insert. Ahora ya podrá pegar lo que desea y no habrá un desplazamiento
del texto indentándose cada vez más. No olvide de quitar luego el modo paste con

\begin{lstlisting}[gobble=2,style=bashinteract,escapechar=!]
  :set nopaste
\end{lstlisting}

Existen varios modos de copiar y pegar cadenas de texto de distintos ficheros en Vi. Lo primero que debe
tener en cuenta es que, para copiar/pegar cosas del/al portapapeles a/desde un terminal\footnote{Aunque aquí
se hable de \emph{terminal}, al estar trabajando en un entorno de escritorio, en realidad quiere decir,
\emph{emulador de terminal}.}, deberá usar una variación de las combinaciones usuales de comandos: deberá usar
también la tecla \tecla{\(\Uparrow\)} siempre que opere en el terminal.  Así, para copiar texto de, por
ejemplo, una web, a su terminal, deberá copiar primero el texto al portapapeles tal y como se hace
normalmente, es decir, con la combinación \tecla{ctrl} \tecla{c} y luego, para copiarlo del portapapeles a su
terminal, deberá usar \tecla{ctrl} \tecla{\(\Uparrow\)} \tecla{v}. En el caso opuesto, es decir, copiar de su
terminal a una aplicación que sí opere directamente con el portapapeles del escritorio normalmente, deberá
usar primero \tecla{ctrl} \tecla{\(\Uparrow\)} \tecla{c} y, luego, \tecla{ctrl} \tecla{v}. Este método sirve
para copiar/pegar cosas de/a su terminal y, por supuesto, también desde Vi, pero al trabajar con Vi tendrá
que tener en cuenta una particularidad: para copiar/pegar mediante este método, deberá estar en modo insert.

Existen también otros métodos más incómodos para copiar y pegar entre ficheros abiertos con un mismo proceso
de Vi (por ejemplo, ficheros abiertos en pestañas distintas de la misma sesión de Vi). Lo que debe hacer es:

\begin{enumerate}
  \item entrar en modo visual (vea \nameref{subsec:vim-sel-text}),
  \item hacer \foreignlanguage{english}{\emph{yank}} al texto seleccionado, cosa que se consigue pulsando la
    tecla \tecla{y},
  \item ir a la ubicación donde desea pegar el texto seleccionado,
  \item pulsar la tecla \tecla{p} (de \foreignlanguage{english}{\emph{paste}}).
\end{enumerate}

\noindent Lo que no se puede conseguir de este modo es pegar algo que hayamos copiado en algo externo a Vi,
como, p.ej., una página web o un documento de LibreOffice. Es decir, no permite copiar de este modo cosas del
portapapeles del escritorio. Tampoco permite copiar y pegar de ficheros abiertos en pestañas distintas de tmux,
pues pertenecen a procesos distintos del sistema.

Otro modo, más complicado en mi opinión, sería el siguiente:

\begin{enumerate}
  \item estás en Vi editando un fichero, que llamaremos aquí \path{file1},
  \item en modo comando, pon: \lstinline+e: file2+, siendo \path{file2} el fichero donde está lo que quieres
    copiar; se te abrirá entonces ese fichero en el editor,
  \item copia entonces lo que quieres como sueles hacerlo (\tecla{6} \tecla{y} \tecla{y}, por ejemplo, si
    quieres copiar 6 líneas desde donde estás),
  \item Pon \lstinline+:b1+ para cambiar al buffer 1, es decir, donde está \path{file1},
  \item coloca el cursor donde quieras pegar el texto y, en modo comando, pulsa \tecla{p}.
\end{enumerate}

Al haber abierto también el otro archivo, es decir, al haber usado dos buffers, puedes cambiar de uno a otro,
como si hubieses habierto un tab nuevo en un navegador web. Se cambia de uno a otro con \tecla{ctrl} + \tecla{w}
\tecla{s} o con \lstinline+:tabn+ (\foreignlanguage{english}{\emph{tab next}}) y \lstinline+:tabp+
(\foreignlanguage{english}{\emph{tab previous}}).

Esta explicación está sacada de
\url{http://stackoverflow.com/questions/4620672/copy-and-paste-content-from-one-file-to-another-file-in-vi}.
También explican otras muchas formas de hacerlo. Creo que con \lstinline-"+y- para copia y \lstinline-"+p- para
pegar también debería poderse hacer, pero a mí no me funciona.
