\section{Reemplazar texto}\label{sec:vim-reempl}
% --------------------------------------------------------------------------------------------------------------
Para reemplazar una determinada cadena de texto \val{cad-vieja} de un fichero, puede hacer lo siguiente, en modo
comando:

\begin{lstlisting}[gobble=2,language=bash,style=bashinteract,escapechar=!]
  :%s/!\val{cad-vieja}!/!\val{cad-nueva}!/!\val{opciones}!
\end{lstlisting}

\noindent \lstinline+%+ indica que se haga para todo el documento. Si desea reemplazar texto desde la posición
actual del cursor en adelante, deberá introducir, en modo comando:

\begin{lstlisting}[gobble=2,language=bash,style=bashinteract,escapechar=!]
  :.,$s/!\val{cad-vieja}!/!\val{cad-nueva}!/!\val{opciones}!
\end{lstlisting}

\noindent Ojo con la \lstinline+s+ al final, que es muy común olvidarla. En cuanto al parámetro \val{opciones},
yo suelo usar \lstinline+g+ y \lstinline+c+ (se pueden poner juntos); con \lstinline+c+ consigo que me pregunte
por cada cambio y, con \lstinline+g+, que se haga para todo el documento, no solo la línea en la que estoy.
Como, al preguntarnos, nos da la opción de abortar, este mismo comando podría usarse también simplemente para
buscar.
