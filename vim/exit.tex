\section{Salir}\label{sec:vim-salir}
% --------------------------------------------------------------------------------------------------------------
Para salir, tendrá que introducir, en modo comando, \lstinline+:q+. El programa no le dejará que salga si ha
realizado cambios en el fichero y no los ha guardado. Puede forzar, en ese caso, la salida sin guardar, con
\lstinline+:q!+, en modo comando, claro. En realidad, no es que se salga de Vim con este comando, sino que se
cierra la pestaña que se estaba editando. Si sólo tenía una pestaña, se cerrará el programa.

Se puede guardar y salir con un solo comando: \lstinline+:wq+. Esto guarda el fichero con el mismo nombre que
tenía y lo cierra.
