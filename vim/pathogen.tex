\section{Pathogen}\label{sec:pathogen}
% --------------------------------------------------------------------------------------------------------------
Pathogen es un plugin de Vim (y NeoVim) que le permitirá instalar otros plugins de forma muy sencilla. Si desea
instalar Pathogen en Vim/NeoVim usando Git, deberá introducir lo siguiente:

\begin{lstlisting}[gobble=2,language=bash,style=bashinteract,escapechar=!]
  !\promptu! @@mkdir -p ~/.vim/autoload ~/.vim/bundle && \ @@
  !\promptu! @@curl -LSso ~/.vim/autoload/pathogen.vim https://tpo.pe/pathogen.vim@@
\end{lstlisting}

Tenga en cuenta que, si usa NeoVim y lo tiene configurado independientemente de Vim (o si sólo tiene NeoVim),
es decir, si sus ficheros y directorios de configuración no están vinculados con los de Vim, deberá cambiar algo
en el listado anterior. En concreto, si usa el esquema XDG y, por tanto, tiene sus ficheros de configuración de
NeoVim bajo \path{~/.config/nvim/} (repito, sin estar vinculados a los de Vim), debería introducir algo así:

\begin{lstlisting}[gobble=2,language=bash,style=bashinteract,escapechar=!]
  !\promptu! @@mkdir -p ~/.config/nvim/autoload ~/.config/nvim/bundle && \ @@
  !\promptu! @@curl -LSso ~/.config/nvim/autoload/pathogen.vim https://tpo.pe/pathogen.vim@@
\end{lstlisting}

Una vez que tenga correctamente instalado Pathogen, puede instalar otros plugins de Vim/NeoVim muy fácilmente.
Por ejemplo, si desea instalar Nerdtree en Vim deberá hacer lo siguiente:

\begin{lstlisting}[gobble=2,language=bash,style=bashinteract,escapechar=!]
  !\promptu! @@cd ~/.vim/bundle@@
  !\promptu! @@git clone https://github.com/scrooloose/nerdtree.git@@
\end{lstlisting}

En NeoVim (si, al igual que antes, usa el esquema de configuración XDG y no tiene sus ficheros de configuración
de NeoVim vinculados a los de Vim),

\begin{lstlisting}[gobble=2,language=bash,style=bashinteract,escapechar=!]
  !\promptu! @@cd ~/.config/nvim/bundle@@
  !\promptu! @@git clone https://github.com/scrooloose/nerdtree.git@@
\end{lstlisting}
