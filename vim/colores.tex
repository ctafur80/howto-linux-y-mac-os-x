\section{Instalación de esquemas de colores}\label{sec:vim-color}
% --------------------------------------------------------------------------------------------------------------
Para instalar nuevos esquemas de colores, puede hacerlo con un gestor de paquetes que use su sistema; por
ejemplo, con Pacman, que es el que trae Arch Linux, se puede instalar el paquete
\lstinline!vim-colorsamplerpack!; con Ubuntu creo que también habrá un paquete, aunque no sé cómo se llama ahora
mismo. Este paquete instala los esquemas de colores más conocidos; luego, usted deberá editar el fichero de
configuración para poder usar realmente alguno de estos esquemas de colores.

Lo normal es que, si los instala de forma global, los ficheros de los esquemas de colores estén en el directorio
\path{/usr/share/vim/vim<vers\_n>/colors/}, donde \val{vers\_n} es el número de versión de Vim que tiene
instalada en su sistema. Creo que no tiene mucho sentido instalarlos para un único usuario; aun así, si desea
hacerlo así, deberá entonces descomprimir el fichero \path{ColorSamplerPack.zip}, que puede descargar de
\url{http://www.vim.org/scripts/script.php?script_id=625}, a \path{/home/<nom\_usuario>/.vim/colors/}; si no
existe dicho directorio, deberá crearlo. Luego, tal y como se dijo antes, deberá activar alguno de los esquemas
de colores en su fichero de configuración (local, en este caso).

En Windows, para instalación local, es igual sólo que deberá descomprimir en \path{C:\Program
Files\Vim\vimfiles} o en \path{C:\Archivos de programa (x86)\Vim\vimfiles} el fichero. Son dos directorios los
que se sobrescriben: \path{colors} y \path{plugins}.
