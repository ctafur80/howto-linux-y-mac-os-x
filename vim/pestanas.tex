\section{Empleo de pestañas}\label{sec:vim-pestanas}
% --------------------------------------------------------------------------------------------------------------
No suelo usar ya pestañas en Vi. Antes lo hacía porque creía que era más cómodo para copiar y pegar de un
fichero a otro, pero la verdad es que es bastante incordio el simple proceso de cambiar de pestaña. Ahora, si
tengo que editar varios ficheros, usaré un Vi en cada pestaña del emulador de terminal o, simplemente, los
editaré con el editor Atom, que suelo usar más ahora.

Para abrir una nueva pestaña (\foreignlanguage{english}{\emph{tab}}) vacía en Vim, lo que debe hacer es
introducir el comando \lstinline!:tabnew!. Esa pestaña se abrirá sin ningún fichero cargado. Para cargarlo o
crearlo se ha de usar \lstinline!:e <ruta-fichero>!, donde \val{ruta-fichero} indica la ruta del fichero a
abrir. Puede pulsar \tecla{tab}, justo tras introcudir \lstinline!:e!, para poder ir navegando con las flechas
por los ficheros en la barra de Vi, si es que la tiene activa. Con la flecha hacia arriba puede subir de
directorio, y la flecha hacia abajo, para bajar. Una vez que tenga abiertas varias pestañas, puede cambiar de
una a otra con el comando \lstinline+:tabn+ (\foreignlanguage{english}{\emph{next tab}}) y \lstinline+:tabp+
(\foreignlanguage{english}{\emph{previous tab}}).

Puede hacer en un solo paso el proceso de abrir una pestaña nueva con el fichero que desee. Se hace,
simplemente, poniendo \lstinline!:tabnew <nom-fichero>!, en modo comando, donde \val{nom-fichero} es el nombre
del fichero que desea abrir. También puede navegar por aquí con el uso de \tecla{tab}. Si \val{nom-fichero} no
es el nombre de un fichero de esa misma carpeta, se creará dicho fichero, siempre y cuando lo guarde desde Vi.
