\section{NeoVim}\label{sec:nvim}
% --------------------------------------------------------------------------------------------------------------
Ahora mismo suelo usar NeoVim, pues creo que lo que pretenden crear es una versión de Vim que, aun manteniendo
la compatibilidad casi al \SI{100}{\percent}, traerá muchas mejoras subyacentes.

Instalar NeoVim es fácil. Por ejemplo, puede usar Brew en Mac OS X:

\begin{lstlisting}[gobble=2,language=bash,style=bashinteract,escapechar=!]
  !\promptu! @@brew install neovim/neovim/neovim@@
\end{lstlisting}

\noindent También se aconseja, tras la instalación, actualizarlo a la última versión:

\begin{lstlisting}[gobble=2,language=bash,style=bashinteract,escapechar=!]
  !\promptu! @@brew update@@
  !\promptu! @@brew upgrade neovim@@
\end{lstlisting}

\noindent o apt-get, en Debian, Ubuntu, etc. Una vez que lo haya instalado, deberá establecer sus directorios de
configuración para NeoVim.

Para usar en NeoVim la configuración de Vim, deberá introducir, en su shell,

\begin{lstlisting}[gobble=2,language=bash,style=bashinteract,escapechar=!]
  !\promptu! @@mkdir -p ${XDG_CONFIG_HOME:=$HOME/.config}@@
  !\promptu! @@ln -s ~/.vim $XDG_CONFIG_HOME/nvim@@
  !\promptu! @@ln -s ~/.vimrc $XDG_CONFIG_HOME/nvim/init.vim@@
\end{lstlisting}

\noindent Como ve, lo que ha hecho es crear varios vínculos simbólicos que apuntan a sus directorios/ficheros de
configuración de Vim. La variable \lstinline!$HOME!, en Mac OS X, será \path{/Users/<usuario>/}; en Linux,
\path{/home/<usuario>/}.

También puede olvidarse de Vim y usar sus propios directorios y ficheros de configuración para NeoVim. Quizás
esto lo haga más adelante.

Todo lo que se explica aquí, puede encontrarlo, con mayor detalle y mejor expresado (aunque en inglés), en el
fichero de ayuda de Vi nvim-intro. Es decir, abra NeoVim y, en modo normal, introduzca
\lstinline!:help nvim-intro!. Los enlaces de dicho fichero se ``pinchan'' con \tecla{ctrl} + \tecla{]}.
