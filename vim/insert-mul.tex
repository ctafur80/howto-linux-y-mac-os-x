\section{Insertado múltiple}\label{sec:vim-ins-mul}
% --------------------------------------------------------------------------------------------------------------
Con la combinación \tecla{ctrl} + \tecla{v}, se pasa al modo visual block, con el cual puede facilitar la tarea
de editar ficheros enormemente. En este modo se pueden seleccionar partes del texto como si de un rectángulo se
tratase.

Una vez que esté en este modo, puede eliminar las primeras \(n\) columnas de los renglones de texto que ha
seleccionado. Esto lo conseguiría pulsando \tecla{s}; elimina dichas columnas y le deja a usted en modo insert.

También puede insertar, si lo desea, lo que quiera a los primeros \(n\) renglones simplemente seleccionando en
modo visual block un rectángulo de texto de estos renglones (da igual el número de columnas que seleccione) y
pulsando luego \tecla{\(\Uparrow\)} + \tecla{i} (es decir, la \emph{i} mayúscula). Ahora debería hacer el
insertado en el primer renglón de lo que quisiera introducir a todos y luego, al pulsar \tecla{esc}, se
aplicaría a todos los que fueron seleccionadas.
