\section{Hacer automáticamente roturas de líneas}\label{sec:vim-l-break}
% --------------------------------------------------------------------------------------------------------------
Para dar cierto formato al texto en Vim, se pueden hacer varias cosas. Si lo que desea es que el propio Vim haga
el salto de línea por usted creando al final de la línea una rotura de línea
(\foreignlanguage{english}{\emph{line break}}) realmente, puede hacerlo simplemente estableciendo la anchura del
texto (\foreignlanguage{english}{\emph{text width}}):

\begin{lstlisting}[gobble=2,style=bashinteract,escapechar=!]
  @@:set tw=!\val{tw}!@@
\end{lstlisting}

\noindent donde \val{tw} es la anchura que desee establecer. Yo, por el ordenador que uso y la resolución que
empleo normalmente para Vim, suelo darle un valor de 112. Podía haber puesto \lstinline+textwidth+ en lugar de
\lstinline+tw+. Si desea recuadrar el texto para producir las roturas de línea porque, por la causa que sea, no
se encuentran cuadradas, puede ver cómo hacerlo en \ref{sec:vim-recuadrar}.
