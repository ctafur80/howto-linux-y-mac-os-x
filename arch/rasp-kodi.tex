\section{Instalar Kodi en un RbPi2}\label{sec:rasp-arch-kodi}
% --------------------------------------------------------------------------------------------------------------
Los documentos que he consultado para la elaboración de este tutorial son
\url{https://wiki.archlinux.org/index.php/Kodi} y
\url{http://blog.kwarf.com/2015/02/arch-and-kodi-on-the-raspberry-pi-2/}.

Tras la instalación y la configuración básica, puede instalar ahora el cliente de BitTorrent Deluge (vea TKTKTK)
o, si lo prefiere, luego. Creo que es mejor que lo haga después.

Edite \path{/boot/config.txt} y descomente \lstinline+disable_overscan=1+ (para que se aproveche más el espacio
de la pantalla durante la reproducción) y, en la última fila, cambie \lstinline+gpu_mem=64+ por
\lstinline+gpu_mem=320+ (para asignar más RAM para la GPU de su RbPi2, cosa bastante conveniente para reproducir
videos). Si lo desea, también puede overclockear su RbPi2 (vea TKTKTK). Luego reinicie.

Ahora debe instalar el paquete \lstinline+kodi-rbp+ y luego arrancar y habilitar el servicio kodi (o, lo que es
lo mismo, kodi.service). Luego, detenga el servicio.

Ahora deberá instalar un servidor NFS (\foreignlanguage{english}{\emph{network filesystem}}). El servidor NFS es
proporcionado por el paquete \lstinline+nfs-utils+; instálelo. Ahora establezca las comparticiones:

\begin{lstlisting}[gobble=2,language=bash,style=bashinteract,escapechar=!]
  !\promptr! @@mkdir -p /srv/nfs/{tv-shows,movies,music}@@
  !\promptr! @@mkdir -p /mnt/{tv-shows,movies,music}@@
  !\promptr! @@mount --bind /mnt/tv-shows /srv/nfs/tv-shows@@
  !\promptr! @@mount --bind /mnt/movies /srv/nfs/movies@@
  !\promptr! @@mount --bind /mnt/music /srv/nfs/music@@
\end{lstlisting}

\noindent y deberá añadir también las entradas correspondientes, para estos montajes, en el fichero
\path{/etc/fstab}, para que no tenga que montarlas manualmente cada vez que inicia su sistema:

\begin{lstlisting}[gobble=2,language=bash,style=bashinteract,escapechar=!]
  ...

  /mnt/tv-shows   /srv/nfs/tv-shows   none  bind  0 0
  /mnt/movies     /srv/nfs/movies     none  bind  0 0
  /mnt/music      /srv/nfs/music      none  bind  0 0
\end{lstlisting}

\noindent En mi caso quedaría algo así:

\begin{lstlisting}[gobble=2,language=bash,style=bashinteract,escapechar=!]
  #
  # /etc/fstab: static file system information
  #
  # <file system>                               <dir>               <type>  <options> <dump>  <pass>
    UUID=1EA2-423D                              /boot               vfat    defaults  0       0
    UUID=ee1f72c5-774d-42fe-b75e-d0a85de734f0   /                   ext4    defaults  0       0
    UUID=6e072cd6-c20c-4bf1-947e-a8e5b9a88d3b   /home               ext4    defaults  0       2
    /mnt/tv-shows                               /srv/nfs/tv-shows   none    bind      0       0
    /mnt/movies                                 /srv/nfs/movies     none    bind      0       0
    /mnt/music                                  /srv/nfs/music      none    bind      0       0
\end{lstlisting}

Comparta el contenido en \path{/etc/exports}:

\begin{lstlisting}[gobble=2,language=bash,style=bashinteract,escapechar=!]
  /srv/nfs            !\val{ip-n-if}!/24(ro,fsid=0,no_subtree_check)
  /srv/nfs/tv-shows   !\val{ip-n-if}!/24(ro,no_subtree_check,insecure)
  /srv/nfs/movies     !\val{ip-n-if}!/24(ro,no_subtree_check,insecure)
  /srv/nfs/music      !\val{ip-n-if}!/24(ro,no_subtree_check,insecure)
\end{lstlisting}

\noindent donde \val{ip-n-if} es la dirección IP de la interfaz de red de su RbPi2 que vaya a usar para
conectarse a su red local (en mi caso, se trata de una interfaz wireless). Como siempre que se hace un cambio al
fichero \path{/etc/exports}, deberá refrescar los exports:

\begin{lstlisting}[gobble=2,language=bash,style=bashinteract,escapechar=!]
  !\promptr! @@exportfs -rav@@
\end{lstlisting}

Arranque y habilite los servicios rpcbind.service y nfs-server.service.

Ahora debe instalar y configurar el servidor MySQL. Para ello, deberá instalar el paquete \lstinline+mariadb+.
Tal y como le dice Pacman al final de la instalación, 

\begin{lstlisting}[gobble=2,language=bash,style=bashinteract,escapechar=!]
  :: You need to initialize the MariaDB data directory prior to starting
     the service. This can be done with mysql_install_db command, e.g.:
     mysql_install_db --user=mysql --basedir=/usr --datadir=/var/lib/mysql
\end{lstlisting}

\noindent Así, pues, deberá ejecutar el mismo comando que se le indica en la última línea; creo que debe
mantener esos mismos valores:

\begin{lstlisting}[gobble=2,language=bash,style=bashinteract,escapechar=!]
  !\promptr! @@mysql_install_db --user=mysql --basedir=/usr --datadir=/var/lib/mysql@@
\end{lstlisting}

Ahora arranque y habilite el servicio mysqld.service.

Ahora deberá introducir lo siguiente:

\begin{lstlisting}[gobble=2,language=bash,style=bashinteract,escapechar=!]
  !\promptr! @@mysql_secure_installation@@
\end{lstlisting}

\noindent En las preguntas que le haga el script, conteste sí (\lstinline!y!) a todo. Ahora introduzca:

\begin{lstlisting}[gobble=2,language=bash,style=bashinteract,escapechar=!]
  !\promptu! @@mysql -u root -p@@
\end{lstlisting}

\noindent Introduzca la contraseña de root de MySQL que asígnó en el primer paso del script anterior. Ahora le
aparecerá una CLI interactiva de MySQL. Introduzca lo siguiente:

\begin{lstlisting}[gobble=2,language=bash,style=bashinteract,escapechar=!]
  MariaDB [(none)]> @@create user 'kodi' identified by 'kodi';@@
  Query OK, 0 rows affected (0.01 sec)
  MariaDB [(none)]> @@grant all on *.* to 'kodi';@@
  Query OK, 0 rows affected (0.00 sec)
  MariaDB [(none)]> @@\q@@
  Bye
\end{lstlisting}

Ahora hay que configurar Kodi para que use la biblioteca MySQL y los exports NFS. Este paso deberá hacerlo en
cada uno de los nodos Kodi; no así los anteriores. Debe comprobar que tiene instalado el paquete
\lstinline+libnfs+ (vea Comprobar TKTKTK); deberá tenerlo instalado, si instaló \lstinline+nfs-utils+. Ahora
deberá crear el fichero \path{/var/lib/kodi/.kodi} con lo siguiente:

\begin{lstlisting}[gobble=2,language=bash,style=bashinteract,escapechar=!]
  <advancedsettings>
    <videodatabase>
      <type>mysql</type>
      <host>!\val{ip-n-if}!</host>
      <port>3306</port>
      <user>kodi</user>
      <pass>kodi</pass>
    </videodatabase>

    <musicdatabase>
      <type>mysql</type>
      <host>!\val{ip-n-if}!</host>
      <port>3306</port>
      <user>kodi</user>
      <pass>kodi</pass>
    </musicdatabase>

    <videolibrary>
      <importwatchedstate>true</importwatchedstate>
      <importresumepoint>true</importresumepoint>
    </videolibrary>
  </advancedsettings>
\end{lstlisting}

\noindent Ahora deberá hacer que kodi sea el dueño del fichero y el grupo predeterminado sea kodi:

\begin{lstlisting}[gobble=2,language=bash,style=bashinteract,escapechar=!]
  !\promptr! @@chown -R kodi:kodi /var/lib/kodi@@
\end{lstlisting}

Para instalar algunos add-ons interesantes, puede seguir el tutorial que hay en
\url{https://www.youtube.com/watch?v=YteNG2cT8uc}.
