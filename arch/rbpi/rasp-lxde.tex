\section{Entorno de escritorio Lxde}\label{sec:rasp-lxde}
% --------------------------------------------------------------------------------------------------------------
Para instalar el entorno de escritorio Lxde en su Raspberry Pi con sistema operativo Arch Linux para ARM, deberá
instalar tres paquetes:

\begin{lstlisting}[gobble=2,language=bash,style=bashinteract,escapechar=!]
  !\promptr! @@pacman -S xf86-video-fbdev lxde xorg-xinit}
\end{lstlisting}

\noindent Luego, quizás deba reiniciar. Después, para ejecutar el entonrno de escritorio, escriba desde su
shell:

\begin{lstlisting}[gobble=2,language=bash,style=bashinteract,escapechar=!]
  !\promptu! @@xinit /usr/bin/lxsession}
\end{lstlisting}

Puede hacer que el servicio \lstinline!lxdm!, que es el servicio de display manager de Lxde, se ejecute al
arrancar el sistema. No tiene que instalarlo, pues se instala junto con el paquete \lstinline!lxde!. Con Systemd
se haría del siguiente modo:

\begin{lstlisting}[gobble=2,language=bash,style=bashinteract,escapechar=!]
  !\promptr! @@systemctl enable lxdm.service}
\end{lstlisting}

\noindent Por si no sabe qué he querido decir con esto último, se podría decir que esto hace que arranque Lxde
al iniciar el sistema.
