\section{Instalación y configuración del wifi}\label{sec:rasp-wifi}
% --------------------------------------------------------------------------------------------------------------
Ahora sería bueno instalar los drivers del chip wifi, si es que le ha puesto algún dongle de wifi. Si desea usar
su sistema para aprender o practicar ataques a redes wifi, deberá entonces usar unos drivers que le permitan
realizar inyección de paquetes, que no son los que se instalan siguiendo este tutorial; este tutorial es para un
uso normal (es decir, simplemente conectarse a su red o a la de una organización). Si le interesa la otra
opción, vea Instalar Software Para Crackear Redes Wi-Fi. Aun así, el paquete compat-wireless, que consta de unos
drivers libres para poder realizar inyección de paquetes con ciertos chips wifi, no soporta aún los chips de
los dongles de wifi que tengo (vea \url{http://www.aircrack-ng.org/doku.php?id=compat-wireless}).

Los dongles que tengo para wi-fi son los siguientes:

\begin{itemize}

  \item Edimax Technology Co., Ltd EW-7811Un 802.11n Wireless Adapter [Realtek RTL8188CUS]. Usa el driver
    rtl8192cu, que ya se instala automáticamente durante la instalación.

  \item 8179 Realtek Semiconductor Corp. RTL8188EUS 802.11n Wireless Network Adapter. Usa el driver r8188eu;
    tampoco tengo que instalarlo.

\end{itemize}

Tal y como explica el wiki de Arch Linux, el proceso de configurar el wireless consta de dos partes:

\begin{enumerate}

  \item Identificar y asegurarse de que en su sistema se encuentra instalado el driver adecuado y configurar la
    interfaz wireless.

  \item Seleccionar un método de gestión de conexiones wireless. El gestor de conexiones que empleo actualmente
    es systemd-networkd.

\end{enumerate}

Lo primero que puede hacer es, una vez logueado, sea conectado realmente o mediante una sesión remota, conectar
a alguno de sus puertos USB el dongle de wifi; no lo haga apagando el sistema e iniciándolo, pues, a mí me ha
pasado que he al hacer esto no se conecta por Ethernet. Ahora vea qué interfaces tiene:

\begin{lstlisting}[gobble=2,language=bash,style=bashinteract,escapechar=!]
  !\promptu! @@networkctl@@
\end{lstlisting}

\noindent Debería aparecerle algo así:

\begin{lstlisting}[gobble=2,language=bash,style=bashinteract,escapechar=!]
  !\promptu! @@networkctl@@
  IDX LINK             TYPE               OPERATIONAL SETUP
    1 lo               loopback           carrier     unmanaged
    2 eth0             ether              routable    configured
    3 wlan0            wlan               off         unmanaged
\end{lstlisting}

\noindent Para obternet información más detallada de alguna interfaz, puede usar:

\begin{lstlisting}[gobble=2,language=bash,style=bashinteract,escapechar=!]
  !\promptu! @@networkctl status !\val{if}!@@
\end{lstlisting}

\noindent Donde \val{if} es el nombre de la interfaz que desea ver. Con este comando puede comprobar, por
ejemplo, que su interfaz wireless tiene instalado un controlador. También le viene bien para ver la dirección
MAC de cada una de sus NIC. Para cambiar ahora el nombre del dispositivo (lo que aparece bajo la columna
\lstinline!LINK! con el comando \lstinline!networkctl!), debe crear una regla udev. Para hacer esto, debe editar
\path{/etc/udev/rules.d/10-network.rules} (y crearlo, si no existe) e introducir lo siguiente:

\begin{lstlisting}[gobble=2,language=bash,style=bashinteract,escapechar=!]
  SUBSYSTEM=="net", ACTION=="add", ATTR{address}=="aa:bb:cc:dd:ee:ff", NAME="en0"
  SUBSYSTEM=="net", ACTION=="add", ATTR{address}=="ff:ee:dd:cc:bb:aa", NAME="wl0"
\end{lstlisting}

\noindent Si la dirección MAC de la interfaz a la que quiere llamar en0 es aa:bb:cc:dd:ee:ff y, la de wl0,
ff:ee:dd:cc:bb:aa (cámbielas por las suyas). Yo suelo seguir la regla de llamar en\val{n}, siendo \val{n} un
número empezando por el 0, a las interfaces Ethernet y wl\val{n}, empezando por el número 0, también, a las
wireless. Tendrá que reiniciar para ver el/los nombre(s) nuevo(s).

MEJOR: Introduzca lo siguiente:
%
%\begin{lstlisting}[gobble=2,language=bash,style=bashinteract,escapechar=!]
%  !\promptr! @@mv /root/10-network.rules /etc/udev/rules.d/@@
%\end{lstlisting}
%
%Ahora,

\begin{lstlisting}[gobble=2,language=bash,style=bashinteract,escapechar=!]
  !\promptr! @@mv /root/rasp/network/{en,wl}0.* /etc/systemd/network/ && \ @@
  > @@rm /etc/systemd/network/eth0.*@@
\end{lstlisting}

Ahora,

\begin{lstlisting}[gobble=2,language=bash,style=bashinteract,escapechar=!]
  !\promptr! @@mv /root/rasp/network/wpa_supplicant-skel.conf /etc/wpa_supplicant/wpa_supplicant-wl0.conf@@
\end{lstlisting}

Ahora,

\begin{lstlisting}[gobble=2,language=bash,style=bashinteract,escapechar=!]
  !\promptr! @@cat /root/rasp/network/wpa_supplicant-zfur.conf >> /etc/wpa_supplicant/wpa_supplicant-wl0.conf@@
\end{lstlisting}

Luego,

\begin{lstlisting}[gobble=2,language=bash,style=bashinteract,escapechar=!]
  !\promptr! @@rm /root/rasp/network/wpa_supplicant-zfur.conf ; rm /etc/wpa_supplicant/wpa_supplicant.conf@@
\end{lstlisting}

Ahora,

\begin{lstlisting}[gobble=2,language=bash,style=bashinteract,escapechar=!]
  !\promptr! @@mv /root/rasp/network/network-wireless@.service /etc/systemd/system/ && \ @@
  > @@mv /root/rasp/network/network-wireless-wl0 /etc/systemd/@@
\end{lstlisting}






















Ahora deberá crear los ficheros de configuración de sus interfaces wifi. Estos ficheros deben encontrarse en
\path{/etc/systemd/network/} y deberán tener como extensión \path{.link} o \path{.network}. Los que yo uso los
puede encontrar en \path{~/Dropbox/Documentos/rasp/<nom_host>/}, donde \path{<nom_host>} es el nombre de host de
mi RbPi2 (tengo configuraciones distintas para los distintos aparatos, puesto que, por ejemplo, cada uno tendrá
direcciones IP distintas). Estos ficheros deberá copiarlos al directorio \path{/etc/systemd/network/} de su
RbPi2. Deberá cambiar también, antes de reiniciar, el fichero de configuración de su interfaz, que se encuentra
en \path{/etc/systemd/network/}. Ahora mismo tiene sólo uno llamado \path{eth0.network}. Deberá borrarlo y
copiar ahí, desde mi ordenador, si no lo ha hecho antes, \path{~/Dropbox/Documentos/rasp/<nom_host>/en0.network}
y \path{~/Dropbox/Documentos/rasp/<nom_host>/en0.link}. Ahora sí puede reiniciar y volver a entrar por SSH sin
problemas.

Una vez hecho esto, deberá instalar WPA Supplicant junto con sus dependencias:

\begin{lstlisting}[gobble=2,language=bash,style=bashinteract,escapechar=!]
  !\promptr! @@pacman -S wpa_supplicant@@
\end{lstlisting}

Ahora deberá copiar el fichero de configuración de WPA Supplicant, llamado, \path{wpa_supplicant-<w-if>.conf},
donde \val{w-if} es el nombre de su interfaz wireless, que puede encontrar en mi ordenador en
\path{~/Dropbox/Documentos/rasp/<nom_host>/}, al directorio \path{/etc/wpa_supplicant/} de su RbPi2; si lo ha
copiado antes a su RbPi2, ya lo tendrá en algún directorio suyo. Tendrá que cambiar las configuraciones de estos
dos últimos ficheros de los que le he hablado, puesto que la contraseña de su wifi será distinta a la mía y las
direcciones que su router asigna a su RbPi2. También suelo borrar el fichero \path{wpa_supplicant.conf} que ya
existe en \path{/etc/wpa_supplicant}.

%Ahora sucede un problema que aún no he sido capaz de solucionar. Si tengo configuradas las interfaces en0 y wl0,
%al desconectar el cable de red Ethernet del RbPi2, no consigo que, tras un reinicio, tenga conexión por wl0.
%Sólo lo consigo si he eliminado los ficheros de configuración de en0. Quizás haya alguna opción para esto.

Ahora es el momento de que cree una unidad de systemd para el demonio de WPA Supplicant. Para ello, puede
copiar, si no lo ha hecho ya, mi fichero \path{~/Dropbox/Documentos/rasp/network-wireless@.service} al
directorio \path{/etc/systemd/system/} de su RbPi2. También deberá crear un fichero donde almacene las variables
de entorno que usa el script. Yo lo creo en  \path{/etc/systemd/} y lo llamo \path{network-wireless-wl0}; esta
ruta tendrá que ponerla en el script, junto a \lstinline!EnvironmentFile!. Este fichero se encuentra también en
mi ordenador, en \path{~/Dropbox/Documentos/rasp/<nom_host>/}. Algo que no soy capaz de solucionar es que, en
\path{network-wireless@wl0.service}, tengo que invocar al comando \lstinline+wpa_supplicant+ necesariamente con
la opción \lstinline+-B+, es decir, para que se ejecute en segundo plano, mientras que en los servicios
predefinidos no habilitados para cargar al inicio que vienen con WPA Supplicant (en
\path{/usr/lib/systemd/system/}) los invocan en primer plano.

Ahora deberá habilitar el demonio \lstinline!network-wireless@\val{w-if}.service!:

\begin{lstlisting}[gobble=2,language=bash,style=bashinteract,escapechar=!]
  !\promptr! @@systemctl enable network-wireless@!\val{w-if}!.service@@
\end{lstlisting}

Ahora debería copiar los ficheros \path{~/Dropbox/Documentos/rasp/<nom_host>/wl0.network} y
\path{~/Dropbox/Documentos/rasp/<nom_host>/wl0.link} a \path{/etc/systemd/network/}. Ahora apague (no reinicie)
su RbPi2. Una vez apagada, desenchufe el cable Ethernet y enciéndala. Ahora debería ser capaz de entrar por SSH
a su RbPi2 por medio de la interfaz \val{w-if}. Puede ver con el comando

\begin{lstlisting}[gobble=2,language=bash,style=bashinteract,escapechar=!]
  !\promptu! @@systemctl --type=service@@
\end{lstlisting}

\noindent que el servicio \path{network-wireless@<w-if>.service} está iniciado, y puede ver el estado de dicho
servicio con:

\begin{lstlisting}[gobble=2,language=bash,style=bashinteract,escapechar=!]
  !\promptu! @@systemctl status network-wireless@!\val{w-if}!.service@@
\end{lstlisting}

Con esto, tendría ya conectada su RbPi2 a una wifi. Cuando lo hice, luego me di cuenta de un problema. Si la
tiene conectada a un monitor o a un televisor, verá que, donde debería aparacer el prompt para loguearse con
alguna cuenta del sistema, le aparece repetidas veces un mensaje que contiene:

\begin{lstlisting}[gobble=2,language=bash,style=bashinteract,escapechar=!]
  cfg80211: Calling CRDA to update world regulatory domain
\end{lstlisting}

\noindent Y, finalmente, tras varias veces, aparecerá un mensaje que dice:

\begin{lstlisting}[gobble=2,language=bash,style=bashinteract,escapechar=!]
  cfg80211: Exceded CRDA call max attempts. Not calling CRDA
\end{lstlisting}

Estos mensajes son mensajes de error que muestra el kernel. Puede verlos también con el comando:

\begin{lstlisting}[gobble=2,language=bash,style=bashinteract,escapechar=!]
  !\promptr! @@dmesg | tail -10@@
\end{lstlisting}

Según leí en el foro oficial de Arch Linux de ARM, el problema se soluciona instalando el paquete
\lstinline!crda! y, posteriormente, descomentando

\begin{lstlisting}[gobble=2,language=bash,style=bashinteract,escapechar=!]
  #WIRELESS_REGDOM="\val{XX}"
\end{lstlisting}

\noindent de \path{/etc/conf.d/wireless-regdom}, donde \val{XX} indica el país o la región de la que desea tener
configurada su interfaz wireless. El valor \lstinline!00! indica world domain; España es \lstinline!ES!. Para
comprobar que lo ha cambiado correctamente, introduzca el comando

\begin{lstlisting}[gobble=2,language=bash,style=bashinteract,escapechar=!]
  !\promptu! @@iw reg get@@
\end{lstlisting}











\begin{comment}
% --------------------------------------------------------------------------------------------------------------





Ahora hay que instalar y habilitar el demonio NetworkManager, pero antes, por si acaso, compruebe que no lo
tiene corriendo ya:

# systemctl --type=service

Este comando muestra todos los servicios que están cargados correctamente. Si no ve uno llamado
NetworkManager.service, introduzca:

# systemctl start NetworkManager

Ahora introduzca:

# nmtui

Con esto, entrará en un interfaz gráfico de consola (sí, esas cosas existen; nmtui quiere decir Network Manager
Terminal User Interface) con el que podrá configurar su acceso a redes, tanto wi-fi como wired. En el menú que
aparece, seleccione Edit a connection. Luego, seleccione el SSID de la red a la que se desea conectar y luego le
pedirá que introduzca la contraseña para conectarse a esa red. Una vez configurada, pinche en Quit y volverá al
shell. TKTKTKTKTKTKTKTKTK

Una vez fuera del TUI, puede volver a usar networkctl para ver si ahora tiene portadora (carrier). Es
aconsejable configurar el router para que le asigne siempre la misma dirección IP a dicha interfaz del sistema
al igual que se dijo para la interfaz eth0, puesto que para entrar por SSH es más cómodo pues no tendré que
buscarla cada vez que desee entrar.

Ahora, para habilitar el demonio NetworkManager para el inicio del sistema, introduzca:

# systemctl enable NetworkManager

Ahora, para comprobar si se ha configurado todo bien, deberá apagar el sistema, desconectar el cable de red
Ethernet y arrancar el sistema (cosa que deberá hacer desconectando el cable que le suministra energía al
RbPi2 y volviéndolo a conectar, pues este aparato no tiene botón de encendido/apagado). Para entrar por
SSH, tenga en cuenta que ahora tendrá una dirección IP distinta, pues su RbPi2 está conectado ahora a
la misma red con otra interfaz, la de su tarjeta wi-fi.

Una cosa rara que sucede a veces tras este paso es que las letras salen muy grandes en la pantalla del televisor
al que tengo conectada la RbPi2.


\end{comment}
