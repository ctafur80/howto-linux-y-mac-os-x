\section{Conectar su RbPi a un monitor VGA}\label{sec:rasp-vga}
% --------------------------------------------------------------------------------------------------------------
Para conectar su RbPi a un monitor que no tenga interfaz HDMI pero si una VGA, puede comprar algún adaptador
compatible (cerciónese de que es compatible con las RbPi) y usarlo, siempre y cuando configure adecuadamente su
RbPi. En \url{https://www.raspberrypi.org/documentation/configuration/config-txt.md} puede leer una explicación
del fichero \path{/boot/config.txt} que tendrá en su RbPi (al menos, si tiene instalado Raspbian o Arch Linux
para ARM), que en las RbPi es el fichero que sustituye a la BIOS de los sistemas de arquitectura x86.

Resumiendo lo que explica esta web, usted podría seguir los pasos siguientes (estamos suponiendo que usted tiene
acceso al sistema de ficheros de su RbPi mediante sesión SSH o montando el ``disco duro'', es decir, la tarjeta
microSD, en otro ordenador):

\begin{enumerate}
  \item Establezca los valores
    \begin{lstlisting}[gobble=6,language=bash,style=bashinteract,escapechar=!]
      hdmi_group=1
      hdmi_mode=1
    \end{lstlisting}
    \noindent en su fichero \path{/boot/config.txt} (compruebe que no están comentados, es decir, que están sin
    \verb+#+ a la izquierda). Guarde el fichero con las modificaciones y reinicie su RbPi.
  \item Introduzca el comando siguiente:
    \begin{lstlisting}[gobble=6,language=bash,style=bashinteract,escapechar=!]
      !\promptr! @@/opt/vc/bin/tvservice -m CEA@@
    \end{lstlisting}
    \noindent y guarde o anote la salida. Tendrá entonces lista de modos CEA soportados por su monitor. Puede
    que uno de los modos venga acompañado de ``native''; ésta es la mejor de las opciones que le muestra.
  \item Introduzca el comando siguiente:
    \begin{lstlisting}[gobble=6,language=bash,style=bashinteract,escapechar=!]
      !\promptr! @@/opt/vc/bin/tvservice -m DMT@@
    \end{lstlisting}
    \noindent y guarde o anote la salida. Tendrá entonces lista de modos DMT soportados por su monitor. Puede
    que uno de los modos venga acompañado de ``native''; ésta es la mejor de las opciones que le muestra.
  \item Introduzca el comando:
    \begin{lstlisting}[gobble=6,language=bash,style=bashinteract,escapechar=!]
      !\promptr! @@/opt/vc/bin/tvservice -s@@
    \end{lstlisting}
    \noindent lo cual le mostrara el estado actual, es decir qué modo está usando actualmente.
  \item Asigne ahora a la clave \texttt{hdmi\_mode} (recuerde, en \path{/boot/config.txt}) el valor que desee de
    los compatibles que se le han mostrado.
\end{enumerate}

Puede que siguiendo estos pasos ya le funcione el monitor. Aun así, deberá comprobar que le funciona no sólo con
un reinicio suave, es decir, con un \lstinline!reboot!, sino que también le funciona con un reinicio fuerte, es
decir, apagando (\lstinline!poweroff!), desconectando la alimentación y volviendo a alimentar. Si le funciona en
ambos casos, ya ha terminado. Si no es así, deberá seguir lo que se indica en el paso siguiente:

\begin{enumerate}[resume]
  \item Introduzca los comandos siguientes para almacenar información más detallada de su monitor:
    \begin{lstlisting}[gobble=6,language=bash,style=bashinteract,escapechar=!]
      !\promptr! @@/opt/vc/bin/tvservice -d /boot/edid.dat@@
      !\promptr! @@/opt/vc/bin/edidparser /boot/edid.dat@@
    \end{lstlisting}
  Añada también, a su fichero \path{/boot/config.txt}, \lstinline!hdmi_edid_file=1!.
\end{enumerate}
