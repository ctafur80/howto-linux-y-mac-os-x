\section{Entorno de escritorio Xfce}\label{sec:rasp-xfce}
% --------------------------------------------------------------------------------------------------------------
En \url{http://blog.adityapatawari.com/2013/05/arch-linux-on-raspberry-pi-running-xfce.html} hay un manual que
está bien, aunque es de 2013.

Lo primero será instalar las librerías Xorg:

\begin{lstlisting}[gobble=2,language=bash,style=bashinteract,escapechar=!]
  !\promptr! @@pacman -S xorg-xinit xorg-server xorg-server-utils xterm@@
\end{lstlisting}

\noindent Ahora, instale Xfce:

\begin{lstlisting}[gobble=2,language=bash,style=bashinteract,escapechar=!]
  !\promptr! @@pacman -S xfce4@@
\end{lstlisting}

\noindent Le preguntará si desea instalar sólo ciertos paquetes. Yo elijo instalar todo, que es la opción
predeterminada. También me gusta instalar un paquete que le proporciona plugins para Xfce4:

\begin{lstlisting}[gobble=2,language=bash,style=bashinteract,escapechar=!]
  !\promptr! @@pacman -S xfce4-goodies@@
\end{lstlisting}

\noindent También instalo los drivers de display:

\begin{lstlisting}[gobble=2,language=bash,style=bashinteract,escapechar=!]
  !\promptr! @@pacman -S xf86-video-fbdev xf86-video-vesa@@
\end{lstlisting}

Ahora, para arrancar el entorno de escritorio XFCE, deberá introducir:

\begin{lstlisting}[gobble=2,language=bash,style=bashinteract,escapechar=!]
  !\promptu! @@startx@@
\end{lstlisting}

\noindent o

\begin{lstlisting}[gobble=2,language=bash,style=bashinteract,escapechar=!]
  !\promptu! @@startxfce4@@
\end{lstlisting}

Si desea arrancarlo al iniciar el sistema, puede hacerlo de dos formas:

\begin{enumerate}

  \item Mediante el empleo de un login manager (en español, se podría decir \emph{gestor de login}). Esto hará
    que aparezca una pantalla en la que podrá seleccionar el usuario, el entorno de escritorio (si es que tiene
    varios instalados), el idioma, etc. Puede elegir de entre varios login managers, independientemente del
    entorno de escritorio, aunque también es cierto que algunos entornos de escritorio tienen su login manager
    predeterminado, que mantiene el mismo diseño y se suelen instalar junto con el entorno de escritorio.
    Existen login managers GUI y CLI. En Xfce4, el login manager predeterminado es Xfwm. Yo no he sabido
    configurarlo, así que he usado LightDM. Para instalar éste último, deberá introducir:

    \begin{lstlisting}[gobble=2,language=bash,style=bashinteract,escapechar=!]
      !\promptr! @@pacman -S lightdm@@
    \end{lstlisting}

    \noindent Luego, deberá instalar un greeter para LightDM:

    \begin{lstlisting}[gobble=2,language=bash,style=bashinteract,escapechar=!]
      !\promptr! @@pacman -S lightdm-gtk-greeter@@
    \end{lstlisting}

    \noindent Ahora debería habilitar el login manager que haya seleccionado (de ahora en adelante,
    \val{l-manager}):

    \begin{lstlisting}[gobble=2,language=bash,style=bashinteract,escapechar=!]
      # systemctl enable \val{l-manager}.service
    \end{lstlisting}

    \noindent Ahora debería funcionar. Si no es así, usted debería reestablecer un vínculo simbólico
    \path{default.target} propio que apunte al \path{graphical.target} predeterminado. Tras habilitar
    \lstinline!\val{l-manager}.service!, debería existir un vínculo simbólico \path{display-manager.service} en
    el directorio \path{/etc/systemd/system/} que apunte a
    \path{/usr/lib/systemd/system/\val{l-manager}.service}.

  \item Añadiendo

    \begin{lstlisting}[gobble=2,language=bash,style=bashinteract,escapechar=!]
      [[ -z $DISPLAY && $XDG_VTNR -eq 1 ]] && exec startx
    \end{lstlisting}

    \noindent a su fichero \path{~/.bash\_profile}. Esto hará que el sistema muestre un prompt en CLI en su
    shell para que acceda a una sesión de una cuenta de usuario de su sistema y, una vez introducida
    correctamente la contraseña, entrará en su shell y directamente el sistema abrirá el entorno de escritorio
    para la misma cuenta de usuario.

\end{enumerate}

\noindent No tiene sentido que configure el inicio de ambas formas. Lo normal es que elija una, preferiblemente
la primera. La segunda tiene más sentido si desea crear ciertos usuarios de su sistema que no tengan acceso al
entorno de escritorio.

A la hora de eliminar Xfce de mi sistema, tuve problemas, pues me decía que tal software no podía eliminarse
porque requería \lstinline!xfce4-panel!. Al final, lo solucioné usando la opción \lstinline!-Rc!, en lugar de
\lstinline!-Rns!, que son las que suelo usar para eliminar software con Pacman. \lstinline!-Rc! es más agresivo
y puede eliminar partes necesarias de su sistema, así que no debe usarse a la ligera.
