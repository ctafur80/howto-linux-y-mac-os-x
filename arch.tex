\chapter{Arch Linux}\label{chapter:arch-linux}
% --------------------------------------------------------------------------------------------------------------
Dedico aquí un capítulo completo a la distribución Arch Linux porque creo que con ésta se puede aprender el
funcionamiento interno de Linux de un modo muy práctico, aunque luego cada sistema tenga muchas
particularidades. La distribución Arch Linux le ``obliga'' a configurar su sistema por usted mismo (en realidad,
si lo desea, también puede usar instaladores que han creado, pero aquí no se usarán). Además, puede instalarla
en sistemas baratos con arquitectura ARM, como, por ejemplo, el Raspberry Pi 2 (debe ser el modelo 2, puesto que
es el que tiene arquitectura ARMv7), y así poder tener un sistema sólo para aprender con el que poder hacer mil
cosas que le vendrán muy bien para aprender sin miedo a ``estropear'' nada. Una vez que aprenda, puede usar esa
máquina con otros propósitos, para que no sienta que ha tirado el dinero: pupede usarla como media center, como
servidor de ficheros CIFS/SMB, etc.

Gran parte de los apartados incluidos aquí hacen referencia al uso del gestor de paquetes Pacman, que es el
predeterminado en Arch Linux.


\input{./arch/update}
\input{./arch/rm-orphan-p}
\input{./arch/reboot}
\input{./arch/tab-completion}
\input{./arch/hostnames}
\input{./arch/timedate}
\input{./arch/grupo-nuevo}
\input{./arch/crear-usuario}
\input{./arch/user-add-groups}
\input{./arch/lang}
\input{./arch/rasp-inst-b}
\input{./arch/rasp-inst-c}
\section{Instalación y configuración del wifi}\label{sec:rasp-wifi}
% --------------------------------------------------------------------------------------------------------------
Ahora sería bueno instalar los drivers del chip wifi, si es que le ha puesto algún dongle de wifi. Si desea usar
su sistema para aprender o practicar ataques a redes wifi, deberá entonces usar unos drivers que le permitan
realizar inyección de paquetes, que no son los que se instalan siguiendo este tutorial; este tutorial es para un
uso normal (es decir, simplemente conectarse a su red o a la de una organización). Si le interesa la otra
opción, vea Instalar Software Para Crackear Redes Wi-Fi. Aun así, el paquete compat-wireless, que consta de unos
drivers libres para poder realizar inyección de paquetes con ciertos chips wifi, no soporta aún los chips de
los dongles de wifi que tengo (vea \url{http://www.aircrack-ng.org/doku.php?id=compat-wireless}).

Los dongles que tengo para wi-fi son los siguientes:

\begin{itemize}

  \item Edimax Technology Co., Ltd EW-7811Un 802.11n Wireless Adapter [Realtek RTL8188CUS]. Usa el driver
    rtl8192cu, que ya se instala automáticamente durante la instalación.

  \item 8179 Realtek Semiconductor Corp. RTL8188EUS 802.11n Wireless Network Adapter. Usa el driver r8188eu;
    tampoco tengo que instalarlo.

\end{itemize}

Tal y como explica el wiki de Arch Linux, el proceso de configurar el wireless consta de dos partes:

\begin{enumerate}

  \item Identificar y asegurarse de que en su sistema se encuentra instalado el driver adecuado y configurar la
    interfaz wireless.

  \item Seleccionar un método de gestión de conexiones wireless. El gestor de conexiones que empleo actualmente
    es systemd-networkd.

\end{enumerate}

Lo primero que puede hacer es, una vez logueado, sea conectado realmente o mediante una sesión remota, conectar
a alguno de sus puertos USB el dongle de wifi; no lo haga apagando el sistema e iniciándolo, pues, a mí me ha
pasado que he al hacer esto no se conecta por Ethernet. Ahora vea qué interfaces tiene:

\begin{lstlisting}[gobble=2,language=bash,style=bashinteract,escapechar=!]
  !\promptu! @@networkctl@@
\end{lstlisting}

\noindent Debería aparecerle algo así:

\begin{lstlisting}[gobble=2,language=bash,style=bashinteract,escapechar=!]
  !\promptu! @@networkctl@@
  IDX LINK             TYPE               OPERATIONAL SETUP
    1 lo               loopback           carrier     unmanaged
    2 eth0             ether              routable    configured
    3 wlan0            wlan               off         unmanaged
\end{lstlisting}

\noindent Para obternet información más detallada de alguna interfaz, puede usar:

\begin{lstlisting}[gobble=2,language=bash,style=bashinteract,escapechar=!]
  !\promptu! @@networkctl status !\val{if}!@@
\end{lstlisting}

\noindent Donde \val{if} es el nombre de la interfaz que desea ver. Con este comando puede comprobar, por
ejemplo, que su interfaz wireless tiene instalado un controlador. También le viene bien para ver la dirección
MAC de cada una de sus NIC. Para cambiar ahora el nombre del dispositivo (lo que aparece bajo la columna
\lstinline!LINK! con el comando \lstinline!networkctl!), debe crear una regla udev. Para hacer esto, debe editar
\path{/etc/udev/rules.d/10-network.rules} (y crearlo, si no existe) e introducir lo siguiente:

\begin{lstlisting}[gobble=2,language=bash,style=bashinteract,escapechar=!]
  SUBSYSTEM=="net", ACTION=="add", ATTR{address}=="aa:bb:cc:dd:ee:ff", NAME="en0"
  SUBSYSTEM=="net", ACTION=="add", ATTR{address}=="ff:ee:dd:cc:bb:aa", NAME="wl0"
\end{lstlisting}

\noindent Si la dirección MAC de la interfaz a la que quiere llamar en0 es aa:bb:cc:dd:ee:ff y, la de wl0,
ff:ee:dd:cc:bb:aa (cámbielas por las suyas). Yo suelo seguir la regla de llamar en\val{n}, siendo \val{n} un
número empezando por el 0, a las interfaces Ethernet y wl\val{n}, empezando por el número 0, también, a las
wireless. Tendrá que reiniciar para ver el/los nombre(s) nuevo(s).

MEJOR: Introduzca lo siguiente:
%
%\begin{lstlisting}[gobble=2,language=bash,style=bashinteract,escapechar=!]
%  !\promptr! @@mv /root/10-network.rules /etc/udev/rules.d/@@
%\end{lstlisting}
%
%Ahora,

\begin{lstlisting}[gobble=2,language=bash,style=bashinteract,escapechar=!]
  !\promptr! @@mv /root/rasp/network/{en,wl}0.* /etc/systemd/network/ && \ @@
  > @@rm /etc/systemd/network/eth0.*@@
\end{lstlisting}

Ahora,

\begin{lstlisting}[gobble=2,language=bash,style=bashinteract,escapechar=!]
  !\promptr! @@mv /root/rasp/network/wpa_supplicant-skel.conf /etc/wpa_supplicant/wpa_supplicant-wl0.conf@@
\end{lstlisting}

Ahora,

\begin{lstlisting}[gobble=2,language=bash,style=bashinteract,escapechar=!]
  !\promptr! @@cat /root/rasp/network/wpa_supplicant-zfur.conf >> /etc/wpa_supplicant/wpa_supplicant-wl0.conf@@
\end{lstlisting}

Luego,

\begin{lstlisting}[gobble=2,language=bash,style=bashinteract,escapechar=!]
  !\promptr! @@rm /root/rasp/network/wpa_supplicant-zfur.conf ; rm /etc/wpa_supplicant/wpa_supplicant.conf@@
\end{lstlisting}

Ahora,

\begin{lstlisting}[gobble=2,language=bash,style=bashinteract,escapechar=!]
  !\promptr! @@mv /root/rasp/network/network-wireless@.service /etc/systemd/system/ && \ @@
  > @@mv /root/rasp/network/network-wireless-wl0 /etc/systemd/@@
\end{lstlisting}






















Ahora deberá crear los ficheros de configuración de sus interfaces wifi. Estos ficheros deben encontrarse en
\path{/etc/systemd/network/} y deberán tener como extensión \path{.link} o \path{.network}. Los que yo uso los
puede encontrar en \path{~/Dropbox/Documentos/rasp/<nom_host>/}, donde \path{<nom_host>} es el nombre de host de
mi RbPi2 (tengo configuraciones distintas para los distintos aparatos, puesto que, por ejemplo, cada uno tendrá
direcciones IP distintas). Estos ficheros deberá copiarlos al directorio \path{/etc/systemd/network/} de su
RbPi2. Deberá cambiar también, antes de reiniciar, el fichero de configuración de su interfaz, que se encuentra
en \path{/etc/systemd/network/}. Ahora mismo tiene sólo uno llamado \path{eth0.network}. Deberá borrarlo y
copiar ahí, desde mi ordenador, si no lo ha hecho antes, \path{~/Dropbox/Documentos/rasp/<nom_host>/en0.network}
y \path{~/Dropbox/Documentos/rasp/<nom_host>/en0.link}. Ahora sí puede reiniciar y volver a entrar por SSH sin
problemas.

Una vez hecho esto, deberá instalar WPA Supplicant junto con sus dependencias:

\begin{lstlisting}[gobble=2,language=bash,style=bashinteract,escapechar=!]
  !\promptr! @@pacman -S wpa_supplicant@@
\end{lstlisting}

Ahora deberá copiar el fichero de configuración de WPA Supplicant, llamado, \path{wpa_supplicant-<w-if>.conf},
donde \val{w-if} es el nombre de su interfaz wireless, que puede encontrar en mi ordenador en
\path{~/Dropbox/Documentos/rasp/<nom_host>/}, al directorio \path{/etc/wpa_supplicant/} de su RbPi2; si lo ha
copiado antes a su RbPi2, ya lo tendrá en algún directorio suyo. Tendrá que cambiar las configuraciones de estos
dos últimos ficheros de los que le he hablado, puesto que la contraseña de su wifi será distinta a la mía y las
direcciones que su router asigna a su RbPi2. También suelo borrar el fichero \path{wpa_supplicant.conf} que ya
existe en \path{/etc/wpa_supplicant}.

%Ahora sucede un problema que aún no he sido capaz de solucionar. Si tengo configuradas las interfaces en0 y wl0,
%al desconectar el cable de red Ethernet del RbPi2, no consigo que, tras un reinicio, tenga conexión por wl0.
%Sólo lo consigo si he eliminado los ficheros de configuración de en0. Quizás haya alguna opción para esto.

Ahora es el momento de que cree una unidad de systemd para el demonio de WPA Supplicant. Para ello, puede
copiar, si no lo ha hecho ya, mi fichero \path{~/Dropbox/Documentos/rasp/network-wireless@.service} al
directorio \path{/etc/systemd/system/} de su RbPi2. También deberá crear un fichero donde almacene las variables
de entorno que usa el script. Yo lo creo en  \path{/etc/systemd/} y lo llamo \path{network-wireless-wl0}; esta
ruta tendrá que ponerla en el script, junto a \lstinline!EnvironmentFile!. Este fichero se encuentra también en
mi ordenador, en \path{~/Dropbox/Documentos/rasp/<nom_host>/}. Algo que no soy capaz de solucionar es que, en
\path{network-wireless@wl0.service}, tengo que invocar al comando \lstinline+wpa_supplicant+ necesariamente con
la opción \lstinline+-B+, es decir, para que se ejecute en segundo plano, mientras que en los servicios
predefinidos no habilitados para cargar al inicio que vienen con WPA Supplicant (en
\path{/usr/lib/systemd/system/}) los invocan en primer plano.

Ahora deberá habilitar el demonio \lstinline!network-wireless@\val{w-if}.service!:

\begin{lstlisting}[gobble=2,language=bash,style=bashinteract,escapechar=!]
  !\promptr! @@systemctl enable network-wireless@!\val{w-if}!.service@@
\end{lstlisting}

Ahora debería copiar los ficheros \path{~/Dropbox/Documentos/rasp/<nom_host>/wl0.network} y
\path{~/Dropbox/Documentos/rasp/<nom_host>/wl0.link} a \path{/etc/systemd/network/}. Ahora apague (no reinicie)
su RbPi2. Una vez apagada, desenchufe el cable Ethernet y enciéndala. Ahora debería ser capaz de entrar por SSH
a su RbPi2 por medio de la interfaz \val{w-if}. Puede ver con el comando

\begin{lstlisting}[gobble=2,language=bash,style=bashinteract,escapechar=!]
  !\promptu! @@systemctl --type=service@@
\end{lstlisting}

\noindent que el servicio \path{network-wireless@<w-if>.service} está iniciado, y puede ver el estado de dicho
servicio con:

\begin{lstlisting}[gobble=2,language=bash,style=bashinteract,escapechar=!]
  !\promptu! @@systemctl status network-wireless@!\val{w-if}!.service@@
\end{lstlisting}

Con esto, tendría ya conectada su RbPi2 a una wifi. Cuando lo hice, luego me di cuenta de un problema. Si la
tiene conectada a un monitor o a un televisor, verá que, donde debería aparacer el prompt para loguearse con
alguna cuenta del sistema, le aparece repetidas veces un mensaje que contiene:

\begin{lstlisting}[gobble=2,language=bash,style=bashinteract,escapechar=!]
  cfg80211: Calling CRDA to update world regulatory domain
\end{lstlisting}

\noindent Y, finalmente, tras varias veces, aparecerá un mensaje que dice:

\begin{lstlisting}[gobble=2,language=bash,style=bashinteract,escapechar=!]
  cfg80211: Exceded CRDA call max attempts. Not calling CRDA
\end{lstlisting}

Estos mensajes son mensajes de error que muestra el kernel. Puede verlos también con el comando:

\begin{lstlisting}[gobble=2,language=bash,style=bashinteract,escapechar=!]
  !\promptr! @@dmesg | tail -10@@
\end{lstlisting}

Según leí en el foro oficial de Arch Linux de ARM, el problema se soluciona instalando el paquete
\lstinline!crda! y, posteriormente, descomentando

\begin{lstlisting}[gobble=2,language=bash,style=bashinteract,escapechar=!]
  #WIRELESS_REGDOM="\val{XX}"
\end{lstlisting}

\noindent de \path{/etc/conf.d/wireless-regdom}, donde \val{XX} indica el país o la región de la que desea tener
configurada su interfaz wireless. El valor \lstinline!00! indica world domain; España es \lstinline!ES!. Para
comprobar que lo ha cambiado correctamente, introduzca el comando

\begin{lstlisting}[gobble=2,language=bash,style=bashinteract,escapechar=!]
  !\promptu! @@iw reg get@@
\end{lstlisting}











\begin{comment}
% --------------------------------------------------------------------------------------------------------------





Ahora hay que instalar y habilitar el demonio NetworkManager, pero antes, por si acaso, compruebe que no lo
tiene corriendo ya:

# systemctl --type=service

Este comando muestra todos los servicios que están cargados correctamente. Si no ve uno llamado
NetworkManager.service, introduzca:

# systemctl start NetworkManager

Ahora introduzca:

# nmtui

Con esto, entrará en un interfaz gráfico de consola (sí, esas cosas existen; nmtui quiere decir Network Manager
Terminal User Interface) con el que podrá configurar su acceso a redes, tanto wi-fi como wired. En el menú que
aparece, seleccione Edit a connection. Luego, seleccione el SSID de la red a la que se desea conectar y luego le
pedirá que introduzca la contraseña para conectarse a esa red. Una vez configurada, pinche en Quit y volverá al
shell. TKTKTKTKTKTKTKTKTK

Una vez fuera del TUI, puede volver a usar networkctl para ver si ahora tiene portadora (carrier). Es
aconsejable configurar el router para que le asigne siempre la misma dirección IP a dicha interfaz del sistema
al igual que se dijo para la interfaz eth0, puesto que para entrar por SSH es más cómodo pues no tendré que
buscarla cada vez que desee entrar.

Ahora, para habilitar el demonio NetworkManager para el inicio del sistema, introduzca:

# systemctl enable NetworkManager

Ahora, para comprobar si se ha configurado todo bien, deberá apagar el sistema, desconectar el cable de red
Ethernet y arrancar el sistema (cosa que deberá hacer desconectando el cable que le suministra energía al
RbPi2 y volviéndolo a conectar, pues este aparato no tiene botón de encendido/apagado). Para entrar por
SSH, tenga en cuenta que ahora tendrá una dirección IP distinta, pues su RbPi2 está conectado ahora a
la misma red con otra interfaz, la de su tarjeta wi-fi.

Una cosa rara que sucede a veces tras este paso es que las letras salen muy grandes en la pantalla del televisor
al que tengo conectada la RbPi2.


\end{comment}

\input{./arch/rasp-bittorrent}
\section{Instalar Kodi en un RbPi2}\label{sec:rasp-arch-kodi}
% --------------------------------------------------------------------------------------------------------------
Los documentos que he consultado para la elaboración de este tutorial son
\url{https://wiki.archlinux.org/index.php/Kodi} y
\url{http://blog.kwarf.com/2015/02/arch-and-kodi-on-the-raspberry-pi-2/}.

Tras la instalación y la configuración básica, puede instalar ahora el cliente de BitTorrent Deluge (vea TKTKTK)
o, si lo prefiere, luego. Creo que es mejor que lo haga después.

Edite \path{/boot/config.txt} y descomente \lstinline+disable_overscan=1+ (para que se aproveche más el espacio
de la pantalla durante la reproducción) y, en la última fila, cambie \lstinline+gpu_mem=64+ por
\lstinline+gpu_mem=320+ (para asignar más RAM para la GPU de su RbPi2, cosa bastante conveniente para reproducir
videos). Si lo desea, también puede overclockear su RbPi2 (vea TKTKTK). Luego reinicie.

Ahora debe instalar el paquete \lstinline+kodi-rbp+ y luego arrancar y habilitar el servicio kodi (o, lo que es
lo mismo, kodi.service). Luego, detenga el servicio.

Ahora deberá instalar un servidor NFS (\foreignlanguage{english}{\emph{network filesystem}}). El servidor NFS es
proporcionado por el paquete \lstinline+nfs-utils+; instálelo. Ahora establezca las comparticiones:

\begin{lstlisting}[gobble=2,language=bash,style=bashinteract,escapechar=!]
  !\promptr! @@mkdir -p /srv/nfs/{tv-shows,movies,music}@@
  !\promptr! @@mkdir -p /mnt/{tv-shows,movies,music}@@
  !\promptr! @@mount --bind /mnt/tv-shows /srv/nfs/tv-shows@@
  !\promptr! @@mount --bind /mnt/movies /srv/nfs/movies@@
  !\promptr! @@mount --bind /mnt/music /srv/nfs/music@@
\end{lstlisting}

\noindent y deberá añadir también las entradas correspondientes, para estos montajes, en el fichero
\path{/etc/fstab}, para que no tenga que montarlas manualmente cada vez que inicia su sistema:

\begin{lstlisting}[gobble=2,language=bash,style=bashinteract,escapechar=!]
  ...

  /mnt/tv-shows   /srv/nfs/tv-shows   none  bind  0 0
  /mnt/movies     /srv/nfs/movies     none  bind  0 0
  /mnt/music      /srv/nfs/music      none  bind  0 0
\end{lstlisting}

\noindent En mi caso quedaría algo así:

\begin{lstlisting}[gobble=2,language=bash,style=bashinteract,escapechar=!]
  #
  # /etc/fstab: static file system information
  #
  # <file system>                               <dir>               <type>  <options> <dump>  <pass>
    UUID=1EA2-423D                              /boot               vfat    defaults  0       0
    UUID=ee1f72c5-774d-42fe-b75e-d0a85de734f0   /                   ext4    defaults  0       0
    UUID=6e072cd6-c20c-4bf1-947e-a8e5b9a88d3b   /home               ext4    defaults  0       2
    /mnt/tv-shows                               /srv/nfs/tv-shows   none    bind      0       0
    /mnt/movies                                 /srv/nfs/movies     none    bind      0       0
    /mnt/music                                  /srv/nfs/music      none    bind      0       0
\end{lstlisting}

Comparta el contenido en \path{/etc/exports}:

\begin{lstlisting}[gobble=2,language=bash,style=bashinteract,escapechar=!]
  /srv/nfs            !\val{ip-n-if}!/24(ro,fsid=0,no_subtree_check)
  /srv/nfs/tv-shows   !\val{ip-n-if}!/24(ro,no_subtree_check,insecure)
  /srv/nfs/movies     !\val{ip-n-if}!/24(ro,no_subtree_check,insecure)
  /srv/nfs/music      !\val{ip-n-if}!/24(ro,no_subtree_check,insecure)
\end{lstlisting}

\noindent donde \val{ip-n-if} es la dirección IP de la interfaz de red de su RbPi2 que vaya a usar para
conectarse a su red local (en mi caso, se trata de una interfaz wireless). Como siempre que se hace un cambio al
fichero \path{/etc/exports}, deberá refrescar los exports:

\begin{lstlisting}[gobble=2,language=bash,style=bashinteract,escapechar=!]
  !\promptr! @@exportfs -rav@@
\end{lstlisting}

Arranque y habilite los servicios rpcbind.service y nfs-server.service.

Ahora debe instalar y configurar el servidor MySQL. Para ello, deberá instalar el paquete \lstinline+mariadb+.
Tal y como le dice Pacman al final de la instalación, 

\begin{lstlisting}[gobble=2,language=bash,style=bashinteract,escapechar=!]
  :: You need to initialize the MariaDB data directory prior to starting
     the service. This can be done with mysql_install_db command, e.g.:
     mysql_install_db --user=mysql --basedir=/usr --datadir=/var/lib/mysql
\end{lstlisting}

\noindent Así, pues, deberá ejecutar el mismo comando que se le indica en la última línea; creo que debe
mantener esos mismos valores:

\begin{lstlisting}[gobble=2,language=bash,style=bashinteract,escapechar=!]
  !\promptr! @@mysql_install_db --user=mysql --basedir=/usr --datadir=/var/lib/mysql@@
\end{lstlisting}

Ahora arranque y habilite el servicio mysqld.service.

Ahora deberá introducir lo siguiente:

\begin{lstlisting}[gobble=2,language=bash,style=bashinteract,escapechar=!]
  !\promptr! @@mysql_secure_installation@@
\end{lstlisting}

\noindent En las preguntas que le haga el script, conteste sí (\lstinline!y!) a todo. Ahora introduzca:

\begin{lstlisting}[gobble=2,language=bash,style=bashinteract,escapechar=!]
  !\promptu! @@mysql -u root -p@@
\end{lstlisting}

\noindent Introduzca la contraseña de root de MySQL que asígnó en el primer paso del script anterior. Ahora le
aparecerá una CLI interactiva de MySQL. Introduzca lo siguiente:

\begin{lstlisting}[gobble=2,language=bash,style=bashinteract,escapechar=!]
  MariaDB [(none)]> @@create user 'kodi' identified by 'kodi';@@
  Query OK, 0 rows affected (0.01 sec)
  MariaDB [(none)]> @@grant all on *.* to 'kodi';@@
  Query OK, 0 rows affected (0.00 sec)
  MariaDB [(none)]> @@\q@@
  Bye
\end{lstlisting}

Ahora hay que configurar Kodi para que use la biblioteca MySQL y los exports NFS. Este paso deberá hacerlo en
cada uno de los nodos Kodi; no así los anteriores. Debe comprobar que tiene instalado el paquete
\lstinline+libnfs+ (vea Comprobar TKTKTK); deberá tenerlo instalado, si instaló \lstinline+nfs-utils+. Ahora
deberá crear el fichero \path{/var/lib/kodi/.kodi} con lo siguiente:

\begin{lstlisting}[gobble=2,language=bash,style=bashinteract,escapechar=!]
  <advancedsettings>
    <videodatabase>
      <type>mysql</type>
      <host>!\val{ip-n-if}!</host>
      <port>3306</port>
      <user>kodi</user>
      <pass>kodi</pass>
    </videodatabase>

    <musicdatabase>
      <type>mysql</type>
      <host>!\val{ip-n-if}!</host>
      <port>3306</port>
      <user>kodi</user>
      <pass>kodi</pass>
    </musicdatabase>

    <videolibrary>
      <importwatchedstate>true</importwatchedstate>
      <importresumepoint>true</importresumepoint>
    </videolibrary>
  </advancedsettings>
\end{lstlisting}

\noindent Ahora deberá hacer que kodi sea el dueño del fichero y el grupo predeterminado sea kodi:

\begin{lstlisting}[gobble=2,language=bash,style=bashinteract,escapechar=!]
  !\promptr! @@chown -R kodi:kodi /var/lib/kodi@@
\end{lstlisting}

Para instalar algunos add-ons interesantes, puede seguir el tutorial que hay en
\url{https://www.youtube.com/watch?v=YteNG2cT8uc}.

\section{Conectar su RbPi a un monitor VGA}\label{sec:rasp-vga}
% --------------------------------------------------------------------------------------------------------------
Para conectar su RbPi a un monitor que no tenga interfaz HDMI pero si una VGA, puede comprar algún adaptador
compatible (cerciónese de que es compatible con las RbPi) y usarlo, siempre y cuando configure adecuadamente su
RbPi. En \url{https://www.raspberrypi.org/documentation/configuration/config-txt.md} puede leer una explicación
del fichero \path{/boot/config.txt} que tendrá en su RbPi (al menos, si tiene instalado Raspbian o Arch Linux
para ARM), que en las RbPi es el fichero que sustituye a la BIOS de los sistemas de arquitectura x86.

Resumiendo lo que explica esta web, usted podría seguir los pasos siguientes (estamos suponiendo que usted tiene
acceso al sistema de ficheros de su RbPi mediante sesión SSH o montando el ``disco duro'', es decir, la tarjeta
microSD, en otro ordenador):

\begin{enumerate}
  \item Establezca los valores
    \begin{lstlisting}[gobble=6,language=bash,style=bashinteract,escapechar=!]
      hdmi_group=1
      hdmi_mode=1
    \end{lstlisting}
    \noindent en su fichero \path{/boot/config.txt} (compruebe que no están comentados, es decir, que están sin
    \verb+#+ a la izquierda). Guarde el fichero con las modificaciones y reinicie su RbPi.
  \item Introduzca el comando siguiente:
    \begin{lstlisting}[gobble=6,language=bash,style=bashinteract,escapechar=!]
      !\promptr! @@/opt/vc/bin/tvservice -m CEA@@
    \end{lstlisting}
    \noindent y guarde o anote la salida. Tendrá entonces lista de modos CEA soportados por su monitor. Puede
    que uno de los modos venga acompañado de ``native''; ésta es la mejor de las opciones que le muestra.
  \item Introduzca el comando siguiente:
    \begin{lstlisting}[gobble=6,language=bash,style=bashinteract,escapechar=!]
      !\promptr! @@/opt/vc/bin/tvservice -m DMT@@
    \end{lstlisting}
    \noindent y guarde o anote la salida. Tendrá entonces lista de modos DMT soportados por su monitor. Puede
    que uno de los modos venga acompañado de ``native''; ésta es la mejor de las opciones que le muestra.
  \item Introduzca el comando:
    \begin{lstlisting}[gobble=6,language=bash,style=bashinteract,escapechar=!]
      !\promptr! @@/opt/vc/bin/tvservice -s@@
    \end{lstlisting}
    \noindent lo cual le mostrara el estado actual, es decir qué modo está usando actualmente.
  \item Asigne ahora a la clave \texttt{hdmi\_mode} (recuerde, en \path{/boot/config.txt}) el valor que desee de
    los compatibles que se le han mostrado.
\end{enumerate}

Puede que siguiendo estos pasos ya le funcione el monitor. Aun así, deberá comprobar que le funciona no sólo con
un reinicio suave, es decir, con un \lstinline!reboot!, sino que también le funciona con un reinicio fuerte, es
decir, apagando (\lstinline!poweroff!), desconectando la alimentación y volviendo a alimentar. Si le funciona en
ambos casos, ya ha terminado. Si no es así, deberá seguir lo que se indica en el paso siguiente:

\begin{enumerate}[resume]
  \item Introduzca los comandos siguientes para almacenar información más detallada de su monitor:
    \begin{lstlisting}[gobble=6,language=bash,style=bashinteract,escapechar=!]
      !\promptr! @@/opt/vc/bin/tvservice -d /boot/edid.dat@@
      !\promptr! @@/opt/vc/bin/edidparser /boot/edid.dat@@
    \end{lstlisting}
  Añada también, a su fichero \path{/boot/config.txt}, \lstinline!hdmi_edid_file=1!.
\end{enumerate}

\section{Entorno de escritorio Lxde}\label{sec:rasp-lxde}
% --------------------------------------------------------------------------------------------------------------
Para instalar el entorno de escritorio Lxde en su Raspberry Pi con sistema operativo Arch Linux para ARM, deberá
instalar tres paquetes:

\begin{lstlisting}[gobble=2,language=bash,style=bashinteract,escapechar=!]
  !\promptr! @@pacman -S xf86-video-fbdev lxde xorg-xinit}
\end{lstlisting}

\noindent Luego, quizás deba reiniciar. Después, para ejecutar el entonrno de escritorio, escriba desde su
shell:

\begin{lstlisting}[gobble=2,language=bash,style=bashinteract,escapechar=!]
  !\promptu! @@xinit /usr/bin/lxsession}
\end{lstlisting}

Puede hacer que el servicio \lstinline!lxdm!, que es el servicio de display manager de Lxde, se ejecute al
arrancar el sistema. No tiene que instalarlo, pues se instala junto con el paquete \lstinline!lxde!. Con Systemd
se haría del siguiente modo:

\begin{lstlisting}[gobble=2,language=bash,style=bashinteract,escapechar=!]
  !\promptr! @@systemctl enable lxdm.service}
\end{lstlisting}

\noindent Por si no sabe qué he querido decir con esto último, se podría decir que esto hace que arranque Lxde
al iniciar el sistema.

\section{Entorno de escritorio Xfce}\label{sec:rasp-xfce}
% --------------------------------------------------------------------------------------------------------------
En \url{http://blog.adityapatawari.com/2013/05/arch-linux-on-raspberry-pi-running-xfce.html} hay un manual que
está bien, aunque es de 2013.

Lo primero será instalar las librerías Xorg:

\begin{lstlisting}[gobble=2,language=bash,style=bashinteract,escapechar=!]
  !\promptr! @@pacman -S xorg-xinit xorg-server xorg-server-utils xterm@@
\end{lstlisting}

\noindent Ahora, instale Xfce:

\begin{lstlisting}[gobble=2,language=bash,style=bashinteract,escapechar=!]
  !\promptr! @@pacman -S xfce4@@
\end{lstlisting}

\noindent Le preguntará si desea instalar sólo ciertos paquetes. Yo elijo instalar todo, que es la opción
predeterminada. También me gusta instalar un paquete que le proporciona plugins para Xfce4:

\begin{lstlisting}[gobble=2,language=bash,style=bashinteract,escapechar=!]
  !\promptr! @@pacman -S xfce4-goodies@@
\end{lstlisting}

\noindent También instalo los drivers de display:

\begin{lstlisting}[gobble=2,language=bash,style=bashinteract,escapechar=!]
  !\promptr! @@pacman -S xf86-video-fbdev xf86-video-vesa@@
\end{lstlisting}

Ahora, para arrancar el entorno de escritorio XFCE, deberá introducir:

\begin{lstlisting}[gobble=2,language=bash,style=bashinteract,escapechar=!]
  !\promptu! @@startx@@
\end{lstlisting}

\noindent o

\begin{lstlisting}[gobble=2,language=bash,style=bashinteract,escapechar=!]
  !\promptu! @@startxfce4@@
\end{lstlisting}

Si desea arrancarlo al iniciar el sistema, puede hacerlo de dos formas:

\begin{enumerate}

  \item Mediante el empleo de un login manager (en español, se podría decir \emph{gestor de login}). Esto hará
    que aparezca una pantalla en la que podrá seleccionar el usuario, el entorno de escritorio (si es que tiene
    varios instalados), el idioma, etc. Puede elegir de entre varios login managers, independientemente del
    entorno de escritorio, aunque también es cierto que algunos entornos de escritorio tienen su login manager
    predeterminado, que mantiene el mismo diseño y se suelen instalar junto con el entorno de escritorio.
    Existen login managers GUI y CLI. En Xfce4, el login manager predeterminado es Xfwm. Yo no he sabido
    configurarlo, así que he usado LightDM. Para instalar éste último, deberá introducir:

    \begin{lstlisting}[gobble=2,language=bash,style=bashinteract,escapechar=!]
      !\promptr! @@pacman -S lightdm@@
    \end{lstlisting}

    \noindent Luego, deberá instalar un greeter para LightDM:

    \begin{lstlisting}[gobble=2,language=bash,style=bashinteract,escapechar=!]
      !\promptr! @@pacman -S lightdm-gtk-greeter@@
    \end{lstlisting}

    \noindent Ahora debería habilitar el login manager que haya seleccionado (de ahora en adelante,
    \val{l-manager}):

    \begin{lstlisting}[gobble=2,language=bash,style=bashinteract,escapechar=!]
      # systemctl enable \val{l-manager}.service
    \end{lstlisting}

    \noindent Ahora debería funcionar. Si no es así, usted debería reestablecer un vínculo simbólico
    \path{default.target} propio que apunte al \path{graphical.target} predeterminado. Tras habilitar
    \lstinline!\val{l-manager}.service!, debería existir un vínculo simbólico \path{display-manager.service} en
    el directorio \path{/etc/systemd/system/} que apunte a
    \path{/usr/lib/systemd/system/\val{l-manager}.service}.

  \item Añadiendo

    \begin{lstlisting}[gobble=2,language=bash,style=bashinteract,escapechar=!]
      [[ -z $DISPLAY && $XDG_VTNR -eq 1 ]] && exec startx
    \end{lstlisting}

    \noindent a su fichero \path{~/.bash\_profile}. Esto hará que el sistema muestre un prompt en CLI en su
    shell para que acceda a una sesión de una cuenta de usuario de su sistema y, una vez introducida
    correctamente la contraseña, entrará en su shell y directamente el sistema abrirá el entorno de escritorio
    para la misma cuenta de usuario.

\end{enumerate}

\noindent No tiene sentido que configure el inicio de ambas formas. Lo normal es que elija una, preferiblemente
la primera. La segunda tiene más sentido si desea crear ciertos usuarios de su sistema que no tengan acceso al
entorno de escritorio.

A la hora de eliminar Xfce de mi sistema, tuve problemas, pues me decía que tal software no podía eliminarse
porque requería \lstinline!xfce4-panel!. Al final, lo solucioné usando la opción \lstinline!-Rc!, en lugar de
\lstinline!-Rns!, que son las que suelo usar para eliminar software con Pacman. \lstinline!-Rc! es más agresivo
y puede eliminar partes necesarias de su sistema, así que no debe usarse a la ligera.

